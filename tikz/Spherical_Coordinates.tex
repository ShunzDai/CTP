\tdplotsetmaincoords{60}{110}

%define polar coordinates for some vector
%TODO: look into using 3d spherical coordinate system
\pgfmathsetmacro{\rvec}{.8}
\pgfmathsetmacro{\thetavec}{30}
\pgfmathsetmacro{\phivec}{60}

%start tikz picture, and use the tdplot_main_coords style to implement the display 
%coordinate transformation provided by 3dplot
\begin{tikzpicture}[scale=5,tdplot_main_coords]

%set up some coordinates 
%-----------------------
\coordinate (O) at (0,0,0);

%determine a coordinate (P) using (r,\theta,\phi) coordinates.  This command
%also determines (Pxy), (Pxz), and (Pyz): the xy-, xz-, and yz-projections
%of the point (P).
%syntax: \tdplotsetcoord{Coordinate name without parentheses}{r}{\theta}{\phi}
\tdplotsetcoord{P}{\rvec}{\thetavec}{\phivec}

%draw figure contents
%--------------------

%draw the main coordinate system axes
\draw[thick,->] (0,0,0) -- (0.8,0,0) node[anchor=north east]{$x$};
\draw[thick,->] (0,0,0) -- (0,0.8,0) node[anchor=north west]{$y$};
\draw[thick,->] (0,0,0) -- (0,0,0.8) node[anchor=south]{$z$};

%draw a vector from origin to point (P) 
\draw[-stealth,color=black] (O) -- (P) node[anchor=east]{$r$};

%draw projection on xy plane, and a connecting line
\draw[dashed, color=black] (O) -- (Pxy);
\draw[dashed, color=black] (P) -- (Pxy);

%draw the angle \phi, and label it
%syntax: \tdplotdrawarc[coordinate frame, draw options]{center point}{r}{angle}{label options}{label}
\tdplotdrawarc{(O)}{0.2}{0}{\phivec}{anchor=north}{$\varphi$}


%set the rotated coordinate system so the x'-y' plane lies within the
%"theta plane" of the main coordinate system
%syntax: \tdplotsetthetaplanecoords{\phi}
\tdplotsetthetaplanecoords{\phivec}

%draw theta arc and label, using rotated coordinate system
\tdplotdrawarc[tdplot_rotated_coords]{(0,0,0)}{0.4}{0}{\thetavec}{anchor=south west}{$\theta$}

%draw some dashed arcs, demonstrating direct arc drawing
\draw[dashed,tdplot_rotated_coords] (\rvec,0,0) arc (0:90:\rvec);
\draw[dashed] (\rvec,0,0) arc (0:90:\rvec);

%set the rotated coordinate definition within display using a translation
%coordinate and Euler angles in the "z(\alpha)y(\beta)z(\gamma)" euler rotation convention
%syntax: \tdplotsetrotatedcoords{\alpha}{\beta}{\gamma}
\tdplotsetrotatedcoords{\phivec}{\thetavec}{0}

%translate the rotated coordinate system
%syntax: \tdplotsetrotatedcoordsorigin{point}
\tdplotsetrotatedcoordsorigin{(P)}

%use the tdplot_rotated_coords style to work in the rotated, translated coordinate frame
\draw[thick,tdplot_rotated_coords,->] (0,0,0) -- (0.2,0,0) node[anchor=north west]{$d\theta$};
\draw[thick,tdplot_rotated_coords,->] (0,0,0) -- (0,0.2,0) node[anchor=west]{$d\varphi$};
\draw[thick,tdplot_rotated_coords,->] (0,0,0) -- (0,0,0.2) node[anchor=south]{$dr$};

\tdplotsetrotatedthetaplanecoords{45}

\end{tikzpicture}