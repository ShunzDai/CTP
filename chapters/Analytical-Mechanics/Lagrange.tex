\section{Lagrange力学}\label{sec:Lagrange}
	在Newton力学体系中,我们用Newton方程研究力学系统的演化.所谓“Newton方程”指的是Newton第二定律
	\begin{equation}\label{eq:f=ma}
		\overrightarrow{F} =m\overrightarrow{a}=\frac{d\overrightarrow{p}}{dt},
	\end{equation}
	我们可以像下面这样修改(\ref{eq:f=ma})得到一个有意思的表述形式.我们假设系统所受的合外力是一个保守力,即对于任意一个有界闭区域$S$,$\overrightarrow{F}$满足
	\begin{equation}\label{eq:intfx}
		\int _{\partial S}F_\mu dx^\mu=0,
	\end{equation}
	其中$F_\mu$是$\overrightarrow{F}$在坐标系下的分量.对于任意保守力,我们可以引入一个标量场$V(x)$,使得$F_\mu=-\partial_\mu V(x)$.通过分析量纲可知,标量场$V(x)$应具有能量的量纲,我们将这个标量场称为\textbf{势场}.
	\begin{remark}
		对(\ref{eq:intfx})用一下Stokes定理可以得到
		$$0=\int_{\partial S}F_\mu dx^\mu=\int_S \partial_\nu F_\mu dx^\nu\wedge dx^\mu=-\int_S\partial_\nu \partial_\mu V(x) dx^\nu\wedge dx^\mu,$$
		可见引入标量场$V(x)$并不改变保守力$\overrightarrow{F}$的性质.
	\end{remark}
	于是,借助标量场$V(x)$我们可以将(\ref{eq:f=ma})改写为
	\begin{equation}\label{eq:eloriginal}
		-\partial_\mu V(x)-\frac{dp_\mu}{dt}=0.
	\end{equation}
	至此,我们的准备工作就差不多了.但你可能认为,将Newton方程从(\ref{eq:f=ma})变到(\ref{eq:eloriginal})好像并没有什么,怎么就“有意思”了?确实是这样,但我们要知道这只是一个开始,我们将在后面看到(\ref{eq:eloriginal})蕴含的优越性.
	%我们注意到(\ref{eq:intfx})中的$\overrightarrow{F}\cdot \overrightarrow{dx}$是一个标量,并且它具有能量的量纲\footnote{读者可以思考一下:为什么这里要强调一个“标量”具有能量的量纲?答案在下一页的侧边栏里.}.于是我们就知道了(\ref{eq:intfx})的含义是“质点围绕闭合路径运动一周时,保守力不会改变质点的机械能”.但我们说(\ref{eq:intfx})这个表述太“物理”了,例如,我可以问:$\overrightarrow{dx}$是什么?是有向线段吗?那有向线段又是什么?是矢量吗?如果是矢量,那它又是哪个线性空间里的矢量?……这些问题并不好回答.我的做法是将$\overrightarrow{F}\cdot \overrightarrow{dx}$
	\subsection{最小作用量原理}
		我们总是看到如“水自发的往低处流”,“光线的传播路径总是用时最少的路径”,或者“各种原子的核外电子排布总使得原子总能量最低”等观测现象.这种现象与数学中的“极值”概念相似,于是我们给出这样一个第一性原理,称为\textbf{最小作用量原理}.
		\begin{definition}
			客体的作用量取极值时,总能导出客体所满足的运动方程.
		\end{definition}
		这个定义云里雾里,什么是作用量?为什么作用量要取极值?我们给出这样的解释.
		
		作用量是这样一个无量纲的泛函
		\begin{equation}\label{eq:action1}
			S=\int ds,
		\end{equation}
		其中$ds$是某个微分式\footnote{$ds$本质上是一个微分形式场,即流形外形式丛的截面.我们这里只讨论$ds$是1微分形式场的情况,即普通的微分式.}.我们可以根据不同的情况分别考虑(\ref{eq:action1})的具体形式.令
		\begin{equation}\label{eq:action2}
			S(x^\mu(t);\dot{x}^\mu(t))=\int ds=\frac{1}{\alpha}\int L(x^\mu(t);\dot{x}^\mu(t))dt,
		\end{equation}
		其中$\alpha$是一个比例系数,$L(x^\mu(t);\dot{x}^\mu(t))$是一个函数,$t$是一个参数,这就是一般情形下的作用量.我们可以为式中的比例系数、函数以及参数赋予物理意义,使其拥有表征物理量的量纲,这样我们就可以使作用量描述物理系统了.例如,当我们设比例系数$\alpha$具有长度的量纲,将函数$L$解释为折射率$L=n(x)$,并将参数$t$诠释为光走的几何路程$t=l$,此时(\ref{eq:action2})就变成了
		$$S=\int ds=\frac{1}{\alpha}\int n(x)dl,$$
		这样构造的作用量描述的就是光程.等等.

		于是我们就解释了什么是作用量.现在,我们回答第二个问题,为什么作用量要取极值?很遗憾的是,我无法回答这个问题.最小作用量原理是一个第一性原理,它无法被证伪.但我们知道,物理学是一门以实验为基础的学科.我们用最小作用量原理可以导出各种系统的运动方程,而这些运动方程预言的物理现象经受住了大量实验的考验.凭借“证有不证无”的观念,如果要证明最小作用量原理是错的,就要求我们拿出一个不满足最小作用量原理预言的实验证据,我们可以拿这一实验证据去证明最小作用量原理的错误,但在此之前,我们将全盘接受这一原理.

		我们求作用量的极值时要使用\textbf{变分}.变分的运算规则几乎与微分完全一致,差异仅在作用对象上:前者作用在泛函上,后者作用在普通函数上.现在我们举一个利用最小作用量原理解决实际问题的例子,顺便看看变分是怎样运算的.
		\begin{example}
			(最速降线问题)求质点的轨迹方程,使得在同一平面上,质点由A点至B点用时最短.
			
			由题可知,我们应该寻找作用量
			\begin{eqnarray*}
				S=\frac{1}{\alpha}\int^B_A dt
			\end{eqnarray*}
			的极值,其中$t$表示时间,比例系数$\alpha$具有时间的量纲.我们设质点轨迹方程为隐函数$$f(x(t),y(t))=0,$$其中$x$表示质点在水平方向上的坐标,$y$表示质点在竖直方向上的坐标.在一个无穷小的时间间隔$dt$内,质点将会移动
			\begin{eqnarray*}
				dl=\sqrt{(dx)^2+(dy)^2}=\sqrt{(dx)^2+\left(\frac{dy}{dx}\right)^2(dx)^2}=\sqrt{1+y'^2}dx,
			\end{eqnarray*}
			其中$y'\equiv dy/dx$.考虑到机械能守恒
			\begin{eqnarray*}
				\frac{1}{2}m\left(\frac{dl}{dt}\right)^2=mgy,
			\end{eqnarray*}
			于是我们得到质点的瞬时速度满足
			\begin{eqnarray*}
				\frac{dl}{dt}=\sqrt{2gy}=\frac{\sqrt{1+y'^2}}{dt}dx,
			\end{eqnarray*}
			即
			\begin{eqnarray*}
				dt=\sqrt{\frac{1+y'^2}{2gy}}dx.
			\end{eqnarray*}
			于是我们以$x$为参数,以$y,y'$为泛函变量构造作用量
			\begin{eqnarray*}
				S(y(x);y'(x))=\frac{1}{\alpha}\int^B_Adt=\frac{1}{\alpha}\int^B_A\sqrt{\frac{1+y'^2}{2gy}}dx.
			\end{eqnarray*}
			现在我们计算$S(y(x);y'(x))$的变分,为了与微分运算$d$区分,我们将变分运算记为$\delta$,则
			\begin{eqnarray*}
				\delta S(y(x);y'(x))=\frac{1}{\alpha}\int^B_A\delta\left(\sqrt{\frac{1+y'^2}{2gy}}\right)dx,
			\end{eqnarray*}
			其中
			\begin{eqnarray*}
				\delta\left(\sqrt{\frac{1+y'^2}{2gy}}\right)&=&\frac{\delta}{\delta y}\left(\sqrt{\frac{1+y'^2}{2gy}}\right)\delta y+\frac{\delta}{\delta y'}\left(\sqrt{\frac{1+y'^2}{2gy}}\right)\delta y'\\
				&=&-\frac{1}{2y}\sqrt{\frac{1+y'^2}{2gy}}\delta y+\frac{y'}{2gy}\left(\frac{1+y'^2}{2gy}\right)^{-\frac{1}{2}}\delta \left(\frac{dy}{dx}\right)\\
				&=&-\frac{1}{2y}\sqrt{\frac{1+y'^2}{2gy}}\delta y+\frac{d}{dx}\left[\frac{y'}{2gy}\left(\frac{1+y'^2}{2gy}\right)^{-\frac{1}{2}}\delta y\right]-\frac{d}{dx}\left[\frac{y'}{2gy}\left(\frac{1+y'^2}{2gy}\right)^{-\frac{1}{2}}\right]\delta y\\
			\end{eqnarray*}
		\end{example}
		