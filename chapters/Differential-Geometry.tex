\part{微分几何}\label{Part:Differential_Geometry}
			\begin{margintable}\vspace{1.4in}\footnotesize
				\begin{tabularx}{\marginparwidth}{|X}
					Section~\ref{sec:Coordinates}. 坐标系\\
					%Section~\ref{sec:LieAlgebra}. 联络与曲率\\
					Section~\ref{sec:Manifold}. 流形\\
				\end{tabularx}
			\end{margintable}
	我并不打算像通常的微分几何教程那样循规蹈矩,而是用一种我认为通俗易懂的描述构造non-Euclidean几何框架.当然,这样做会损失一些数学上的严谨性,但我认为这对于物理工作者而言无伤大雅.

	首先,我们将引入描述欧式空间的各种正交曲线坐标系,并推广这一概念得到一般的曲线坐标系,并且通过这些曲线坐标系我们可以诱导出一些重要的线性空间结构.当我们考察一般的弯曲空间(称为流形)时,我们认为弯曲空间的局部与欧式空间等价,这意味着我们可以在流形的每个局部都放入一个坐标系,并用这些坐标系覆盖整个流形.这些坐标系的相交部分通过坐标变换进行参数的转移,这个转移函数构成了Jacobi矩阵.同样的,由坐标系诱导的那些线性空间也通过Jacobi矩阵相互变换.
	