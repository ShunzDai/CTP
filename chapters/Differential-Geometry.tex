\part{微分几何}\label{Part:Differential_Geometry}
	\begin{margintable}\vspace{1.4in}\footnotesize
		\begin{tabularx}{\marginparwidth}{|X}
			Section~\ref{sec:Coordinates}. 局部坐标系\\
			%Section~\ref{sec:LieAlgebra}. 联络与曲率\\
			%Section~\ref{sec:Manifold}. 流形\\
		\end{tabularx}
	\end{margintable}
	内地的物理学工作者喜好在网络社区上整理自己的笔记.这些笔记虽然名义上被称作“笔记”,但实质上都只是抄书罢了.笔者不喜好抄书,也不喜好做笔记,但对于本章节的内容,笔者仍将大量参考Chern的讲义\upcite{bib:Chern01},并斗胆将这本书的所述内容改写成通俗的、适于家用的语言.当然,这样做会损失一些数学上的严谨,但笔者认为这无伤大雅.

	\section{局部坐标系}\label{sec:Coordinates}
微分几何的研究对象是一个称为\textbf{流形}(Manifold)的集合,其上拥有线性结构与自然的拓扑结构.为了引出流形,我们先回顾一下线性空间、度量空间以及拓扑空间的公理化构造,然后逐步往集合上添加这些结构.
	\begin{definition}
		域$F$上的线性空间$V$是这样一个集合,对任意$\alpha,\beta,\gamma\in V$;$a,b\in F$有

		矢量加法映射$V\times V\rightarrow V$:\\
		1)(交换律)$\alpha+\beta=\beta+\alpha$;\\
		2)(结合律)$(\alpha+\beta)+\gamma=\alpha+(\beta+\gamma)$;\\
		3)(零元)存在唯一的$0\in V$,使得$0+\alpha=\alpha$;\\
		4)(逆元)对任意$\alpha\in V$,存在唯一的$\beta\in V$,使得$\alpha+\beta=0$.
		
		矢量数乘映射$F\times V\rightarrow V$:\\
		5)(酉性)对$1\in F$,有$1\alpha=\alpha$;\\
		6)(结合律)$a(b\alpha)=(ab)\alpha$;\\
		7)(分配律1)$(a+b)\alpha=a\alpha+b\alpha$;\\
		8)(分配律2)$a(\alpha+\beta)=a\alpha+a\beta$.
	\end{definition}
	我们用$\mathbb{R}$表示实数域,记$\mathbb{R}^n$表示全体n元有序实数组构成的集合.任意$x\in\mathbb{R}^n$的第$i$个坐标均可用实数$x^i$表示,其中$i=1,\cdots,n$,称为\textbf{抽象指标}.
	\begin{remark}
		需要注意的是,坐标$x^i$的上指标的标记方法是习惯上的约定,不可随意变更为下指标.我们会在稍晚些时候看出,这是非常有效的的符号表征方法.
	\end{remark}
	$\mathbb{R}^n$除了上述的线性构造,还具有自然的度量结构.
	\begin{definition}
		$S$是一个集合,若存在一个映射$d:S\times S\rightarrow \mathbb{R}$,使得对于任意$x,y\in S$都满足:

		1)(正定性)$d(x,y)\geqslant 0$,且$d(x,y)=0$当且仅当$x=y$时成立;\\
		2)(对称性)$d(x,y)=d(y,x)$;\\
		3)(三角不等式)$d(x,z)+d(z,y)\geqslant d(x,y)$,

		则称$(S,d)$是一个度量空间,映射$d$称为$S$上的度量.
	\end{definition}
	对任意$x,y\in \mathbb{R}^n$,命
	\begin{equation}\label{eq:metric of R}
		\begin{split}
			d:\mathbb{R}^n\times \mathbb{R}^n\rightarrow R,(x,y)\mapsto d(x,y)=\sqrt{\sum_{i=1}^n(x^i-y^i)^2},
		\end{split}
	\end{equation}
	容易验证映射(\ref{eq:metric of R})满足度量的定义,于是$(\mathbb{R}^n,d)$是一个度量空间.
	\begin{remark}
		需要注意的是,此时我们仅引入了度量结构(而不是内积结构),所以此时只有“距离”的概念,尚且没有“角度”的概念.
	\end{remark}
	可以发现,$\mathbb{R}^n$还拥有自然的拓扑结构.
	\begin{definition}
		设$S$是一个集合,$O$是一些$S$的子集构成的集合.若$O$满足:

		1)$\varnothing\in O$且$S\in O$;\\
		2)$O$中任意多个元素的并集仍是$O$的元素;\\
		3)$O$中有限多个元素的交集仍是$O$的元素,

		则称$(S,O)$是一个拓扑空间,$O$的元素称为开集.
	\end{definition}
	在$\mathbb{R}^n$中,记半径为$r$的开球为$$O(x,r)=\left\{y\in\mathbb{R}^n:d(x,y)<r\right\},$$若命
	$$O=\left\{O(x,r):x\in\mathbb{R}^n,r>0\right\},$$
	则$(\mathbb{R}^n,O)$成为一个拓扑空间.我们陆续为集合$\mathbb{R}^n$加上线性结构、度量结构、拓扑结构后,$\mathbb{R}^n$便称为Euclidean空间.
	\begin{remark}
		这是Chern的表述.但笔者认为这样构造的$\mathbb{R}^n$上还没有内积结构,似乎并不能被称为“欧氏空间”.笔者认为Chern这样构造是想通过$\mathbb{R}^n$的拓扑结构与流形的拓扑结构对应.欧氏空间的一个比较通俗的定义是直接往线性结构上附加内积结构,不必有拓扑结构.
	\end{remark}
	拓扑空间上还可以加入额外的Hausdorff性质,这种性质强调了拓扑空间是无限可分的.
	\begin{definition}
		设$(S,O)$是一个拓扑空间,若对于任意两点$x,y\in S$,都存在邻域$O(x,a),O(y,b)\in O$,使得$O(x,a)\cap O(y,b)=\varnothing$,则称这个拓扑空间是Hausdorff空间.
	\end{definition}
	然后便可以给出流形的定义了.
	\begin{definition}
		设$(M,O)$是一个Hausdorff空间.若对于任意一点$x\in M$,都存在邻域$O(x,r)\in O$同胚于$\mathbb{R}^n$的一个开集,则称$M$是一个$n$维\textbf{拓扑流形}.
	\end{definition}
	\begin{remark}
		“同胚”指的是一个映射$\phi_{O(x,r)}:O(x,r)\rightarrow U\subset \mathbb{R}^n$,形象的说,就是将弯曲空间的局部与欧氏空间等同起来,建立一个一一对应的映射关系.我们将$(O(x,r),\phi_{O(x,r)})$称为流形$M$的一个\textbf{坐标卡}.
	\end{remark}
	

	%\section{由局部坐标系诱导的线性空间}
    欧式坐标诱导出两种特殊的线性空间,分别称为切空间和余切空间.我们可以定义线性空间与线性空间的“积”,从而构造出更大的线性空间,称为张量空间.另外,全反对称张量作为张量空间的特例具有特别的代数结构,称为外代数.本节将会一一讲述这些由欧式坐标诱导的线性空间.
    \subsection{切空间}
        \begin{definition}
            $\mathbb{R}^n$中的点$p$在坐标系$(x)$下的坐标为$(x^1,\cdots,x^n)$,则$(x^1,\cdots,x^n)$在$p$点诱导了一个线性空间$T_p\mathbb{R}^n$,$T_p\mathbb{R}^n$的元素即导数算符.
        \end{definition}
        我们将$T_p\mathbb{R}^n$称为切空间,$T_p\mathbb{R}^n$的元素称为切矢量.这个定义是说,任意一个切向量$X\in T_p\mathbb{R}^n$都能通过延坐标线的导数算符$\partial_\mu$在$p$点展开为
        \begin{equation}\label{eq:tangent 1}
            X=\left(X^\mu\frac{\partial}{\partial x^\mu}\right)\Bigg|_p,
        \end{equation}
        其中$x^\mu$表示切向量$X$在基底$\partial/\partial x^\mu$下的分量.我们也可以用矩阵显式写出(\ref{eq:tangent 1}),即
        \begin{equation}\label{eq:tangent 2}
            X=
            \begin{pmatrix}
                X^1&\cdots&X^n	
            \end{pmatrix}
            \begin{pmatrix}
                \frac{\partial}{\partial x^1}\\
                \vdots\\
                \frac{\partial}{\partial x^n}	
            \end{pmatrix}.
        \end{equation}
        在进行坐标变换时,切空间的基底像下面这样改变:
        \begin{equation}\label{eq:tangent 3}
            \frac{\partial}{\partial x^\mu}=\frac{\partial x'^\nu}{\partial x^\mu}\frac{\partial}{\partial x'^\nu},
        \end{equation}
        写成矩阵显式就是
        \begin{equation}\label{eq:tangent 4}
            \begin{pmatrix}
                \frac{\partial}{\partial x^1}\\
                \vdots\\
                \frac{\partial}{\partial x^n}	
            \end{pmatrix}
            =
            \begin{pmatrix}
                \frac{\partial x'^1}{\partial x^1}&\cdots&\frac{\partial x'^n}{\partial x^1}\\
                \vdots&\ddots&\vdots\\
                \frac{\partial x'^1}{\partial x^n}&\cdots&\frac{\partial x'^n}{\partial x^n}\\	
            \end{pmatrix}
            \begin{pmatrix}
                \frac{\partial}{\partial x'^1}\\
                \vdots\\
                \frac{\partial}{\partial x'^n}	
            \end{pmatrix}.
        \end{equation}
        我们将(\ref{eq:tangent 4})中产生的$n$阶方阵称为Jocobi矩阵,记为$J$.
        \begin{example}
            计算球坐标系下的切空间基底
            $\left(\frac{\partial}{\partial r}\ \frac{\partial}{\partial \theta}\ \frac{\partial}{\partial \varphi}\right)$
            转换到直角坐标系下的基底的变换关系.
            \begin{eqnarray*}
                \frac{\partial}{\partial r}&=&\frac{\partial x}{\partial r}\frac{\partial }{\partial x}+\frac{\partial y}{\partial r}\frac{\partial }{\partial y}+\frac{\partial z}{\partial r}\frac{\partial }{\partial z}=\sin\theta \cos\varphi\frac{\partial }{\partial x}+\sin\theta \sin\varphi\frac{\partial }{\partial y}+\cos\theta\frac{\partial }{\partial z};\\
                \frac{\partial}{\partial \theta}&=&\frac{\partial x}{\partial \theta}\frac{\partial }{\partial x}+\frac{\partial y}{\partial \theta}\frac{\partial }{\partial y}+\frac{\partial z}{\partial \theta}\frac{\partial }{\partial z}=r\cos\theta \cos\varphi\frac{\partial }{\partial x}+r\cos\theta \sin\varphi\frac{\partial }{\partial y}-r\sin\theta\frac{\partial }{\partial z};\\
                \frac{\partial}{\partial \varphi}&=&\frac{\partial x}{\partial \varphi}\frac{\partial }{\partial x}+\frac{\partial y}{\partial \varphi}\frac{\partial }{\partial y}+\frac{\partial z}{\partial \varphi}\frac{\partial }{\partial z}=-r\sin\theta \sin\varphi\frac{\partial }{\partial x}+r\sin\theta \cos\varphi\frac{\partial }{\partial y}.
            \end{eqnarray*}
            写成矩阵形式就是
            \begin{equation}
                \begin{pmatrix}
                    \frac{\partial}{\partial r}\\
                    \frac{\partial}{\partial \theta}\\
                    \frac{\partial}{\partial \varphi}	
                \end{pmatrix}
                =
                \begin{pmatrix}
                    \sin\theta \cos\varphi&\sin\theta \sin\varphi&\cos\theta\\
                    r\cos\theta \cos\varphi&r\cos\theta \sin\varphi&-r\sin\theta\\
                    -r\sin\theta \sin\varphi&r\sin\theta \cos\varphi&0\\	
                \end{pmatrix}
                \begin{pmatrix}
                    \frac{\partial}{\partial x}\\
                    \frac{\partial}{\partial y}\\
                    \frac{\partial}{\partial z}	
                \end{pmatrix}.
            \end{equation}
        \end{example}
        \begin{exercise}
            计算柱坐标系下的切空间基底
            $\left(\frac{\partial}{\partial r}\ \frac{\partial}{\partial \theta}\ \frac{\partial}{\partial z}\right)$
            转换到直角坐标系下的基底的变换关系.
        \end{exercise}
    \subsection{余切空间}
        \begin{definition}
            $\mathbb{R}^n$中的点$p$在坐标系$(x)$下的坐标为$(x^1,\cdots,x^n)$,则$(x^1,\cdots,x^n)$在$p$点诱导了一个线性空间$T_p^*\mathbb{R}^n$,$T_p^*\mathbb{R}^n$的元素即1微分形式.
        \end{definition}
        我们将$T_p^*\mathbb{R}^n$称为余切空间,$T_p^*\mathbb{R}^n$的元素称为余切矢量.这个定义是说,任意一个余切矢量$\omega\in T_p^*\mathbb{R}^n$都能通过延坐标线的微元$dx^\mu$在$p$点展开为
	%\section{什么是场}
	%\section{流形}\label{sec:Manifold}
	一个流形指的是这样一个结构,其在局部上是平直的,但整体不必是平直的.
	\subsection{Riemannian流形}
	%\input{chapters/Differential-Geometry/Connection.tex}
	\begin{thebibliography}{9}%宽度9
	\bibitem{bib:Chern01}陈省身,陈维桓.微分几何讲义[M].第二版.北京:北京大学出版社,2001
	\end{thebibliography}