\part{引言}
    我从2019年9月就开始写这份笔记了,但写到今天也只有寥寥数页。原本的笔记也有这样一个引言部分,主要讲述了我整个大学四年的艰难历程,但现在看来,那些文字只不过是自己感动自己罢了\footnote{我仍然保留了那些文字,放在我的博客中:\href{https://shunzdai.now.sh/2020/06/04/My-Life-In-College/}{\textit{我的大学}}。}。现在,我重写了这份引言,用尽量简短的语言来阐述这份笔记做了些什么工作。
	
    这份笔记受微分几何的影响颇深,微分几何也是我大学四年以来尤其喜爱的数学领域。相比于线性代数,微分几何为代数问题赋予了几何的看法;相比于微积分(当然,我指的是高等数学中的微积分,而不是数学分析中的微积分),微分几何更注重“基本结构”,它没有繁多的运算技巧,不必将问题拘谨在冗长的计算中。事实上,理论物理与微分几何的关系可以说是“殊途同归”,它们都用到了公理化(或者说是“第一性原理”)的方法。在研究这种问题时,我们总可以从几条基本公理(或基本假设)出发,逐步推理出宏伟的结构。
    
    在繁琐冗长的计算训练中,我们逐步忘却了那些基本结构的作用。当然,这些论断不意味着我们在学习理论物理时不需要任何实际计算、应用,只是说,我们在学习理论物理时更应该将注意力放在基本结构上,而简单的计算能够加深理解,这就已经足够了,我们不必将精力放在如同上述第一个问题的复杂计算上。

    这份笔记中,我们将通篇使用Einstein求和约定,即相同的一对上下指标表示求和。我们将始终保持上下指标的差异,这种差异原则上与度量张量升降指标无关,而是一种源自事物本身的“对偶”属性。

    希望诸位能喜欢上本教程。
    
    $\quad$

    $\quad$

    \rightline{代顺治}
    \rightline{2020年10月31日,于深圳}
    