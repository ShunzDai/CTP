\part{引言}
    笔者从2019年9月就开始写这份文档了,但写到今天也只有寥寥数页。原始版本的文档也有这样一个引言部分,主要讲述了笔者整个大学四年的艰难历程,但现在看来,那些文字只不过是自己感动自己罢了\footnote{笔者仍然保留了那些文字,放在笔者的博客中:\href{https://shunzdai.now.sh/2020/06/04/My-Life-In-College/}{\textit{我的大学}}。}。现在,笔者重写了这份引言,用尽量简短的语言来阐述这份文档做了些什么工作。
	
    这份文档受微分几何的影响颇深,微分几何也是笔者大学四年以来尤其喜爱的数学领域。相比于线性代数,微分几何为代数问题赋予了几何的看法;相比于微积分(当然,笔者指的是高等数学中的微积分,而不是数学分析中的微积分),微分几何更注重“基本结构”,它没有繁多的运算技巧,不必将问题拘谨在冗长的计算中。事实上,理论物理与微分几何的关系可以说是“殊途同归”,它们都用到了公理化(或者说是“第一性原理”)的方法。在研究这种问题时,我们总可以从几条基本公理(或基本假设)出发,逐步推理出宏伟的结构。
    
    在繁琐冗长的计算训练中,我们逐步忘却了那些基本结构的作用。当然,这些论断不意味着我们在学习理论物理时不需要任何实际的计算训练,只是说,我们在学习理论物理时更应该将注意力放在基本结构上,而简单的计算能够加深理解,这就已经足够了。

    在正文中,时常可以看到笔者做的评注(Remark),这些评注将作为正文的补充说明,进一步给出笔者主观的思考.评注里有时也会咬文嚼字,笔者认为做一些训诂也未尝不可。

    希望诸位能喜欢本文。
    
    $\quad$

    $\quad$

    \rightline{代顺治}
    \rightline{2020年10月31日,于深圳}
    