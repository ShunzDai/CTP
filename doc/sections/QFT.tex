\part{量子场论}\label{Part:QFT}
	\begin{margintable}\vspace{1.4in}\footnotesize
		\begin{tabularx}{\marginparwidth}{|X}
		Section~\ref{sec:QM}. 量子力学\\
		\end{tabularx}
	\end{margintable}
	测试测试测试测试测试测试测试测试测试测试测试测试测试测试测试测试测试测试测试测试测试测试测试测试测试测试测试测试测试
	\section{量子力学}\label{sec:QM}
		
		笼统的说,将一个经典物理系统进行量子化的方法有三种.
		
		第一种方法是利用波动力学进行量子化,这种方法的核心是Schrodinger方程
		\begin{equation}
			-\frac{\hbar^2}{2m}\nabla^2\psi(\overrightarrow{x},t)+V(\overrightarrow{x})\psi(\overrightarrow{x},t)=i\hbar\frac{\partial}{\partial t}\psi(\overrightarrow{x},t).
		\end{equation}
		如你所见,Schrodinger方程是一个复的二阶偏微分方程,整个波动力学围绕这个方程展开,不停的解方程,这一过程用到的那些特殊函数我至今没有弄懂.但我可以不负责任的说,整个量子力学的核心不是微分方程,而是代数.因此,接下来的内容与你会不会球谐函数、Legendre函数无关紧要.

		第二种方法是利用矩阵力学进行量子化,这种方法来源于Hamilton力学.之前,我们在Hamilton力学中引入了一个Lie代数结构,称为Poisson括号.通过计算我们发现,整个Ha-
		milton力学都能改写成Poisson括号的形式,体现出和谐的统一性.使用这种方法,我们能对量子力学进行公理化构造,得出非常好的结论.

		至于第三种方法,我们先卖个关子,等到\ref{sec:#}再进行阐述.

		现在,我们给出量子力学的5条公理,接下来的内容完全建立在这5条公理之上.我们要声明,这5条公理不可证伪,因此不存在“为什么是这样”的问题.物理归根结底是一门以实验为基础的学科,以这5条公理为基础建立的整个量子力学体系经受了住大量实验的验证,因此我们说这5条公理是有效的.
		\begin{definition}
			给定一个孤立的力学系统,可根据如下5条公理构建一个量子力学系统:
			\begin{enumerate}
				\item 存在一个复Hilbert空间$\mathcal{H}$,量子系统的状态由$\mathcal{H}$中的矢量$\vert\psi\rangle$	描述,两个矢量$\vert\psi\rangle ,c\vert\psi\rangle(c\in\mathbb{C})$	描述相同状态.		
				\item 经典力学中的力学量$A(\overrightarrow{x} ,t)$对应了$\mathcal{H}$中的一个Hermitian算子
				$$\hat{A}(\overrightarrow{x} ,t):\mathcal{H}\rightarrow \mathcal{H},\hat{A}=\hat{A}^\dagger.$$
				对$A$的测量值是$\hat{A}$的特征值.
				\item 力学量$A,B$的Poisson括号对应了Hermitian算符$\hat{A},\hat{B}$的对易括号
				$$\{A,B\}\Rightarrow\frac{1}{i\hbar}[\hat{A},\hat{B}],$$
				其中$[\hat{A},\hat{B}]=\hat{A}\hat{B}-\hat{B}\hat{A}$.
				\item 若某一时刻量子系统处于$\vert\psi\rangle$状态,我们记此时测量力学量$A$的期望为$\langle A\rangle$,它满足
				$$\langle A\rangle =\frac{\langle\psi\vert\hat{A}\vert\psi\rangle}{\langle\psi\vert\psi\rangle}.$$
				\item 豆腐脑要吃甜的.
			\end{enumerate}
		\end{definition}
		\begin{remark}\ 
			\begin{enumerate}
				\item 完备的内积空间即Hilbert空间.其中完备性要求空间中的任意Cauchy序列均收敛,内积结构使得空间拥有了度量.
				
				$\mathcal{H}$存在一个对偶空间$\mathcal{H}^*$,$\mathcal{H}^*$中的矢量记为$\langle\psi\vert$.设算子$\hat{A}$的一组特征值$\{a_n\}$对应的正交归一特征向量为$\vert n\rangle $,满足$\langle m\vert n\rangle=\delta_{mn}$,其中由$\langle\cdot \vert\cdot\rangle:\mathcal{H}\times \mathcal{H}\rightarrow \mathbb{C} $诱导的内积结构就是$\mathcal{H}$上的内积.
				\item 算子$\hat{A}$的Hermitian性使得$\hat{A}$的特征值必定是实数.这是因为
				\begin{eqnarray*}
					\hat{A}\vert\psi_n\rangle=a_n\vert\psi_n\rangle&\Rightarrow&\langle\psi_n\vert\hat{A}\vert\psi_n\rangle=a_n\langle\psi_n\vert\psi_n\rangle;\\
					\langle\psi_n\vert\hat{A}^\dagger=a^*_n\langle\psi_n\vert&\Rightarrow&\langle\psi_n\vert\hat{A}^\dagger\vert\psi_n\rangle=a^*_n\langle\psi_n\vert\psi_n\rangle,
				\end{eqnarray*}
				考虑到$\hat{A}$的Hermitian性,于是有
				$$a^*_n\langle\psi_n\vert\psi_n\rangle=\langle\psi_n\vert\hat{A}^\dagger\vert\psi_n\rangle=\langle\psi_n\vert\hat{A}\vert\psi_n\rangle=a_n\langle\psi_n\vert\psi_n\rangle,$$
				即$a^*_n=a_n$.要使等式成立,则$a_n$必为实数.这样约定是因为量子系统中出现的测量值为复数时没有意义.
				\item 由(\ref{eq:xxppxp})得
				\begin{eqnarray*}
					\{x^\mu,x^\nu\}&=&0\Rightarrow\frac{1}{i\hbar}[\hat{x}^\mu,\hat{x}^\nu]=0;\\
					\{p_\mu,p_\nu\}&=&0\Rightarrow\frac{1}{i\hbar}[\hat{p}_\mu,\hat{p}_\nu]=0;\\
					\{x^\mu,p_\nu\}&=&\delta^\mu_\nu\Rightarrow\frac{1}{i\hbar}[\hat{x}^\mu,\hat{p}_\nu]=\delta^\mu_\nu.
				\end{eqnarray*}
				\item 若系统的状态矢量$\vert\psi\rangle$是归一化的,则$\langle\psi\vert\psi\rangle=1$,此时$\langle A\rangle =\langle\psi\vert\hat{A}\vert\psi\rangle$.
				\item 豆腐脑不可以是咸的.
			\end{enumerate}
		\end{remark}

		接下来我们便可以开始正题了.
		\subsection{绘景}
			首先,我们给出一个描述力学量算符随时间演化的方程,称为Heisenberg方程.
			\begin{definition}Heisenberg方程是对(\ref{eq:heq1})量子化的结果
				\begin{equation}\label{eq:heq2}
					\frac{dA}{dt}=\{A,H\}\Rightarrow\frac{d\hat{A}}{dt}=\frac{1}{i\hbar}[\hat{A},\hat{H}].
				\end{equation}
			\end{definition}
			(\ref{eq:heq2})是Hamilton力学在量子力学公理下的直接结论,它的解具有特别的结构.
			\begin{lemma}
				(\ref{eq:heq2})的解形如
				\begin{equation}\label{eq:heq'ssolution}
					\hat{A}(t)=e^{-\frac{1}{i\hbar}\hat{H}t}\hat{A}(0)e^{\frac{1}{i\hbar}\hat{H}t}.
				\end{equation}
				
			\end{lemma}
			\begin{proof}
				\begin{eqnarray*}
					\frac{d}{d t}\left(e^{-\frac{1}{i \hbar} \hat{H} t} \hat{A}(0) e^{\frac{1}{i \hbar} \hat{H} t}\right)&=&\frac{d}{d t}\left(e^{-\frac{1}{i \hbar}\hat{H} t}\right)\hat{A}(0) e^{\frac{1}{i \hbar} \hat{H} t}+e^{-\frac{1}{i \hbar} \hat{H} t} \hat{A}(0) \frac{d}{d t}\left(e^{\frac{1}{i \hbar} \hat{H} t}\right)\\
					&=&-\frac{1}{i \hbar} \hat{H} e^{-\frac{1}{i \hbar} \hat{H} t} \hat{A}(0) e^{\frac{1}{i \hbar} \hat{H} t}+e^{-\frac{1}{i \hbar} \hat{H} t} \hat{A}(0) \frac{1}{i \hbar} \hat{H} e^{\frac{1}{i \hbar} \hat{H} t} \\
					&=&-\frac{1}{i \hbar} \hat{H} \hat{A}(t)+\frac{1}{i \hbar} \hat{A}(t) \hat{H}=\frac{1}{i \hbar}[\hat{A}, \hat{H}].
				\end{eqnarray*}
			\end{proof}
			\begin{remark}
				为简化记号,我们命
			\begin{eqnarray*}
				U(t)&\equiv&e^{\frac{1}{i \hbar} \hat{H} t};\\
				U^\dagger (t)&\equiv&e^{-\frac{1}{i \hbar} \hat{H} t},
			\end{eqnarray*}
			于是(\ref{eq:heq'ssolution})可以写成
			\begin{equation}\label{eq:heq'ssolution2}
				\hat{A}(t)=U^\dagger(t)\hat{A}(0)U(t).
			\end{equation}
			\end{remark}
			
			我们将Heisenberg方程描述的体系称为Heisenberg绘景.在Heisenberg绘景下,算符含时,态不含时.我们设系统的归一化的状态为$\vert\psi\rangle_H$,则力学量$A(t)$的期望
			\begin{eqnarray}
				\langle A\rangle&=&\langle\psi\vert_H\hat{A}(t)\vert\psi\rangle_H\notag\\
				&=&\langle\psi\vert_H U^\dagger(t)\hat{A}(0)U(t)\vert\psi\rangle_H\notag\\
				&=&\langle\psi(t)\vert_S\hat{A}(0)\vert\psi(t)\rangle_S,
			\end{eqnarray}
			其中
			\begin{equation}\label{eq:transformation}
				\vert\psi(t)\rangle_S=U(t)\vert\psi\rangle_H,
			\end{equation}
			这个操作称为绘景变换——它将Heisenberg绘景下的状态矢量$\vert\psi\rangle_H$变换为Schrodinger绘景下的状态矢量$\vert\psi(t)\rangle_S$.在Schrodinger绘景下,算符不含时,态含时.我们将(\ref{eq:transformation})两侧同时对时间求导,于是得到
			\begin{equation}\label{eq:seq1}
				\frac{\partial}{\partial t}|\psi\rangle_{S}=\frac{\partial}{\partial t} e^{\frac{1}{i \hbar} \hat{H} t}|\psi\rangle_{H}=\frac{1}{i \hbar} \hat{H} e^{\frac{1}{i \hbar} \hat{H} t}|\psi\rangle_{H}=\frac{1}{i\hbar} \hat{H}|\psi\rangle_{S},
			\end{equation}
			(\ref{eq:seq1})即Schrodinger方程
			\begin{definition}Schrodinger方程为
				\begin{equation}\label{eq:seq2}
					\hat{H}|\psi\rangle_{S}=i\hbar\frac{\partial}{\partial t}|\psi\rangle_{S}.
				\end{equation}
			\end{definition}



\newpage