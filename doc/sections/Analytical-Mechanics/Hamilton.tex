\section{Hamilton力学}\label{sec:Hamilton}
		Lagrange力学描述的是位形流形上的力学规律,是比Newton力学更深刻的分析工具.然而,Lagrange力学也有不足:Euler-Lagrange方程是一个二阶方程,它在大多数情况下不易求解.如果能想办法让描述力学系统的方程的阶数降下来,则在求解运动规律时就方便多了.Hamilton力学由此而生.Hamilton力学是辛流形上的力学规律,其参数为广义坐标$x^\mu$以及广义动量$p_\mu$.Lagrange力学中有Lagrange函数$L(x^\mu;{\dot{x}}^\mu)$,对应的,Hamilton力学中也有Hamilton函数,我们将其表示为$H(x^\mu;p_\mu)$.由Hamilton函数逐步搭建的力学体系能提供丰富的信息,事实上,由Heisenberg建立的矩阵力学就来源于经典Hamilton力学的量子化.
	\subsection{Hamilton方程}
		对Lagrange函数$L(x^\mu;{\dot{x}}^\mu)$变分,得到
		\begin{equation}\label{eq:delta L}
		\delta L=\frac{\partial L}{\partial x^\mu}\delta x^\mu+\frac{\partial L}{\partial {\dot{x}}^\mu}\delta {\dot{x}}^\mu,
		\end{equation}
		Lagrange函数具有能量的量纲,于是Lagrange函数对广义速度的偏导数具有动量的量纲.不妨将这个具有动量量纲的量称为“广义动量”.
		\begin{definition}
		广义动量$p_\mu$定义为
		\begin{equation}
			p_\mu:=\frac{\partial L}{\partial {\dot{x}}^\mu}.
		\end{equation}
		\end{definition}
		于是(\ref{eq:delta L})可以改写成
		\begin{equation}\label{eq:delta L'}
		\delta L=\frac{\partial L}{\partial x^\mu}\delta x^\mu+p_\mu\delta {\dot{x}}^\mu;
		\end{equation}
		再根据Euler-Lagrange方程得到
		\begin{equation*}
		\frac{\partial L}{\partial x^\mu}=\frac{d}{dt}\frac{\partial L}{\partial {\dot{x}}^\mu}={\dot{p}}_\mu,
		\end{equation*}
		所以(\ref{eq:delta L'})变成
		\begin{equation}\label{eq:delta L''}
		\delta L={\dot{p}}_\mu\delta x^\mu+p_\mu\delta {\dot{x}}^\mu.
		\end{equation}
		为了得到某个以$p_\mu$为变量的变分,我们对(\ref{eq:delta L''})进行Legendre变换得到
		\begin{equation*}
		\delta L={\dot{p}}_\mu\delta x^\mu+\delta(p_\mu{\dot{x}}^\mu)-{\dot{x}}^\mu\delta p_\mu,
		\end{equation*}
		移项得到
		\begin{equation}\label{eq:delta H}
		\delta(p_\mu{\dot{x}}^\mu-L)=-{\dot{p}}_\mu\delta x^\mu+{\dot{x}}^\mu\delta p_\mu.
		\end{equation}
		所以,若
		\begin{definition}
		定义Hamilton函数为
		\begin{equation}
			H:=p_\mu{\dot{x}}^\mu-L,
		\end{equation}
		\end{definition}
		则$H$就是以$x^\mu$和$p_\mu$为变量的函数.于是我们可以建立如下原理
		\begin{theorem}
		Hamilton方程组是指如下两个方程
		\begin{equation}
			\begin{split}\label{eq:H}
				\frac{\partial H}{\partial x^\mu}&=-{\dot{p}}_\mu;\\
				\frac{\partial H}{\partial p_\mu}&={\dot{x}}^\mu.
			\end{split}
		\end{equation}
		\end{theorem}
		可见,Hamilton力学研究的是1阶方程,但方程的数量比Lagrange力学增加了一倍.
	
	\subsection{Poisson括号}
		设任意力学量$A$是$x^\mu$与$p_\mu$的函数$A(x^\mu(t);p_\mu(t))$,我们来研究$A$随时间变化的规律.由于
		\begin{equation}\label{eq:Poisson}
		\frac{dA}{dt}=\frac{\partial A}{\partial x^\mu}\frac{d x^\mu}{dt}+\frac{\partial A}{\partial p_\mu}\frac{dp_\mu}{dt},
		\end{equation}
		考虑到Hamilton方程(\ref{eq:H}),带入(\ref{eq:Poisson})得到
		\begin{equation}\label{eq:Poisson'}
		\frac{dA}{dt}=\frac{\partial A}{\partial x^\mu}\frac{\partial H}{\partial p_\mu}-\frac{\partial A}{\partial p_\mu}\frac{\partial H}{\partial x^\mu}.
		\end{equation}
		(\ref{eq:Poisson'})存在着一种反对称性,我们可以用一个双线性映射表示之.
		\begin{definition}
		以$x^\mu$与$p_\mu$为变量的力学量$X$与$Y$的Poisson括号为
		\begin{equation}\label{eq:PB}
			\{X,Y\}:=\frac{\partial X}{\partial x^\mu}\frac{\partial Y}{\partial p_\mu}-\frac{\partial X}{\partial p_\mu}\frac{\partial Y}{\partial x^\mu}.
		\end{equation}
		\end{definition}
		\begin{remark}
			Poisson括号满足
			\begin{enumerate}
				\item 反对称性:$\{X,Y\}=-\{Y,X\}$;
				\item Jacobi恒等式:$\{\{X,Y\},Z\}+\{\{Z,X\}Y\}+\{\{Y,Z\},X\}=0$.
			\end{enumerate}
			事实上Poisson括号构成一个\textbf{Lie代数}.
		\end{remark}
		
		(\ref{eq:Poisson'})可用Poisson括号表示为
		\begin{equation}\label{eq:heq1}
		\frac{dA}{dt}=\{A,H\}.
		\end{equation}
		Hamilton方程也可以修改成Poisson括号的形式
		\begin{equation}
		\begin{split}
			\{p_\mu,H\}&={\dot{p}_\mu};\\
			\{x^\mu,H\}&={\dot{x}}^\mu,
		\end{split}
		\end{equation}
		而$x^\mu$与$x^\nu$、$p_\mu$与$p_\nu$、$x^\mu$与$p_\nu$的Poisson括号分别为
		\begin{eqnarray}\label{eq:xxppxp}
			\{x^\mu,x^\nu\}&=&\frac{\partial x^\mu}{\partial x^\rho}\frac{\partial x^\nu}{\partial p_\rho}-\frac{\partial x^\mu}{\partial p_\rho}\frac{\partial x^\nu}{\partial x^\rho}=0;\\
			\{p_\mu,p_\nu\}&=&\frac{\partial p_\mu}{\partial x^\rho}\frac{\partial p_\nu}{\partial p_\rho}-\frac{\partial p_\mu}{\partial p_\rho}\frac{\partial p_\nu}{\partial x^\rho}=0;\\
			\{x^\mu,p_\nu\}&=&\frac{\partial x^\mu}{\partial x^\rho}\frac{\partial p_\nu}{\partial p_\rho}-\frac{\partial x^\mu}{\partial p_\rho}\frac{\partial p_\nu}{\partial x^\rho}=\delta^\mu_\rho\delta^\rho_\nu=\delta^\mu_\nu.
		\end{eqnarray}
		作为广义动量的特例,我们来看看角动量的Poisson括号.
		角动量即$\overrightarrow{L}=\overrightarrow{r}\times\overrightarrow{p}$,其分量满足
		\begin{equation*}
			L^\rho=\varepsilon^{\rho\mu\nu}x_\mu p_\nu;\\
			L_\rho=\varepsilon_{\rho\mu\nu}x^\mu p^\nu.
		\end{equation*}
		其中$x_\mu\equiv\delta_{\mu\sigma}x^\sigma$,$p^\nu\equiv\delta^{\nu\sigma}p_\sigma$.于是角动量的Poisson括号为
		\begin{eqnarray*}
			\{L_\mu,L^\nu\}&=&\frac{\partial L_\mu}{\partial x^\rho}\frac{\partial L^\nu}{\partial p_\rho}-\frac{\partial L_\mu}{\partial p_\rho}\frac{\partial L^\nu}{\partial x^\rho}\\
			&=&\varepsilon_{\mu\rho\sigma}\varepsilon^{\nu\lambda\rho}p^\sigma x_\lambda-\varepsilon_{\mu\lambda\rho}\varepsilon^{\nu\rho\sigma}x^\lambda p_\sigma\\
			&=&-\delta^\nu_\mu p^\sigma x_\sigma+p^\nu x_\mu+\delta^\nu_\mu x^\sigma p_\sigma-x^\nu p_\mu\\
			&=&\delta^\nu_\mu\left(x^\sigma p_\sigma-p^\sigma x_\sigma\right)+p^\nu x_\mu-x^\nu p_\mu\\
			&=&p^\nu x_\mu-x^\nu p_\mu,
		\end{eqnarray*}
		即$\{L_\mu,L_\nu\}=p_\nu x_\mu-x_\nu p_\mu$.注意到
		$$p_\nu x_\mu-x_\nu p_\mu=\left(\delta^\beta_\nu\delta^\alpha_\mu-\delta^\alpha_\nu\delta^\beta_\mu\right)x_\alpha p_\beta=\varepsilon_{\sigma\mu\nu}\varepsilon^{\sigma\alpha\beta} x_\alpha p_\beta=\varepsilon_{\sigma\mu\nu}L^\sigma,$$
		于是得到
		\begin{equation}\label{eq:[L,L]}
			\{L_\mu,L_\nu\}=\varepsilon_{\sigma\mu\nu}L^\sigma.
		\end{equation}
		我们还可以证明:
		\begin{exercise}
			定义角动量的平方为$L^2\equiv L_\mu L^\mu$,则$L^2$与角动量的Poisson括号为
			\begin{equation}\label{eq:[L^2,L]}
				\{L^2,L_\mu\}=0.
			\end{equation}
		\end{exercise}
		回到力学量$A$随时间变化的规律问题上,我们得到如下引理.
		\begin{lemma}
		若力学量$A(x^\mu;p_\mu)$与Hamiltonian的Poisson括号为零,即
		\begin{equation}
			\{A,H\}=0,
		\end{equation}
		则物理量$A$是一个\textbf{守恒荷}.
		\end{lemma}
		我们举一个例子.
		\begin{example}
			求自由粒子的守恒荷.

		设局部坐标下某自由粒子的Lagrangian为
		$$L=\frac{1}{2}g_{\mu\nu}\dot{x}^\mu\dot{x}^\nu,$$
		则广义动量
		\begin{eqnarray*}
			p_\rho=\frac{\partial L}{\partial {\dot{x}}^\rho}=\frac{1}{2}g_{\mu\nu}\left(\frac{\partial \dot{x}^\mu}{\partial {\dot{x}}^\rho}\dot{x}^\nu+\dot{x}^\mu\frac{\partial \dot{x}^\nu}{\partial {\dot{x}}^\rho}\right)=\frac{1}{2}g_{\mu\nu}\left(\delta^\mu_\rho\dot{x}^\nu+\dot{x}^\mu\delta^\nu_\rho\right)=g_{\rho\mu}\dot{x}^\mu.
		\end{eqnarray*}
		因此$\dot{x}^\mu\equiv g^{\mu\nu}p_\nu$,于是Hamiltonian为
		\begin{eqnarray*}
			H=p_\mu{\dot{x}}^\mu-L=g^{\mu\nu}p_\mu p_\nu-\frac{1}{2}g^{\mu\nu}p_\mu p_\nu=\frac{1}{2}g^{\mu\nu}p_\mu p_\nu,
		\end{eqnarray*}
		用Poisson括号可得
		\begin{eqnarray*}
			\{p_\mu,H\}=0;\\
			\{H,H\}=0.
		\end{eqnarray*}	
		所以此时广义动量和Hamiltonian守恒.
		\end{example}
	
	
	\subsection{Noether定理}
		\begin{theorem}
		设$\epsilon$是一个无穷小量.若无穷小坐标变换$x^\mu\rightarrow x'^\mu=x^\mu+\epsilon f^\mu(x)$保持Hamiltonian不变,则$Q\equiv p_\mu f^\mu(x)$是一个守恒荷.
		\end{theorem}
		\begin{proof}
		要证$Q$是守恒荷,即证$\dot{Q}$为零:
		\begin{equation*}
			\begin{split}
				\dot{Q}&\equiv\frac{dQ}{dt}=\{Q,H\}\\
				&=\frac{\partial Q}{\partial x^\rho}\frac{\partial H}{\partial p_\rho}-\frac{\partial H}{\partial x^\rho}\frac{\partial Q}{\partial p_\rho}\\
				&=p_\mu \frac{\partial f^\mu}{\partial x^\rho}\frac{\partial H}{\partial p_\rho}-f^\rho\frac{\partial H}{\partial x^\rho}.
			\end{split}
		\end{equation*}
		下面来计算Hamiltonian的变分.由于坐标变换保持Hamiltonian不变,则有$\delta H=0$,于是
		\begin{equation}\label{pr:delta H}
			0=\delta H=\frac{\partial H}{\partial x^\mu}\delta x^\mu+\frac{\partial H}{\partial p_\mu}\delta p_\mu,
		\end{equation}
		其中$\delta x^\mu$可直接由坐标变换得到:
		\begin{equation}\label{pr:delta x}
			\begin{split}
				x^\mu\rightarrow x'^\mu&=x^\mu+\epsilon f^\mu(x)\\
				\Rightarrow\delta x^\mu&=\epsilon f^\mu(x);
			\end{split}
		\end{equation}
		$\delta p_\mu$要从广义动量的定义出发得到:
		\begin{eqnarray*}
			&&p_\mu=\frac{\partial L}{\partial \dot{x}^\mu}\\
			\Rightarrow &&p'_\mu=\frac{\partial L}{\partial \dot{x}'^\mu}=\frac{\partial L}{\partial \dot{x}^\nu}\frac{\partial \dot{x}^\nu}{\partial \dot{x}'^\mu}=p_\nu\frac{\partial \dot{x}^\nu}{\partial \dot{x}'^\mu},
		\end{eqnarray*}
		由坐标变换可推得
		\begin{eqnarray*}
			&&x'^\mu=x^\mu+\epsilon f^\mu(x)\\
			\Rightarrow&&\dot{x}'^\mu=\dot{x}^\mu+\epsilon \frac{\partial f^\mu}{\partial x^\rho}\dot{x}^\rho\\
			\Rightarrow&&\frac{\partial \dot{x}'^\mu}{\partial \dot{x}^\nu}=\delta^\mu_\nu+\epsilon \frac{\partial f^\mu}{\partial x^\rho}\delta^\rho_\nu=\delta^\mu_\nu+\epsilon \frac{\partial f^\mu}{\partial x^\nu}\\
			\Rightarrow&&\frac{\partial \dot{x}^\nu}{\partial \dot{x}'^\mu}=\delta^\nu_\mu-\epsilon \frac{\partial f^\nu}{\partial x^\mu},
		\end{eqnarray*}
		其中最后一步可用反证法证明
		\begin{eqnarray*}
			\frac{\partial \dot{x}'^\mu}{\partial \dot{x}^\nu}\frac{\partial \dot{x}^\nu}{\partial \dot{x}'^\rho}=\left(\delta^\mu_\nu+\epsilon \frac{\partial f^\mu}{\partial x^\nu}\right)\left(\delta^\nu_\mu-\epsilon \frac{\partial f^\nu}{\partial x^\mu}\right)=\delta^\mu_\rho+O(\epsilon^2),
		\end{eqnarray*}
		省略高阶无穷小只剩单位矩阵,说明左侧两个矩阵确实是互逆的.但我们尚未证明逆矩阵的唯一性,为此,我们假设
		\begin{eqnarray*}
			\frac{\partial \dot{x}^\nu}{\partial \dot{x}'^\mu}=\delta^\nu_\mu+\epsilon h^\nu_\mu,
		\end{eqnarray*}
		其中$h^\nu_\mu$是以坐标为变量的函数.于是
		\begin{eqnarray*}
			\delta^\mu_\rho&&=\frac{\partial \dot{x}'^\mu}{\partial \dot{x}^\nu}\frac{\partial \dot{x}^\nu}{\partial \dot{x}'^\rho}\\
			&&=\left(\delta^\mu_\nu+\epsilon \frac{\partial f^\mu}{\partial x^\nu}\right)\left(\delta^\nu_\rho+\epsilon h^\nu_\rho\right)\\
			&&=\delta^\mu_\rho+\epsilon\left(\frac{\partial f^\mu}{\partial x^\rho}+h^\mu_\rho\right)+O(\epsilon^2)\\
			&&\Rightarrow\epsilon\left(\frac{\partial f^\mu}{\partial x^\rho}+h^\mu_\rho\right)=0,
		\end{eqnarray*}
		要使上式对于任意无穷小量$\epsilon$都成立,唯有	
		\begin{eqnarray*}
			h^\mu_\rho=-\frac{\partial f^\mu}{\partial x^\rho},
		\end{eqnarray*}
		这样就证明了解的唯一性.所以
		\begin{equation}\label{pr:delta p}
		\begin{split}
			p'_\mu&=p_\nu\frac{\partial \dot{x}^\nu}{\partial \dot{x}'^\mu}=p_\nu\left(\delta^\nu_\mu-\epsilon \frac{\partial f^\nu}{\partial x^\mu}\right)\\
			&=p_\mu-\epsilon p_\nu\frac{\partial f^\nu}{\partial x^\mu}\\
			\Rightarrow\delta p_\mu&=-\epsilon p_\nu\frac{\partial f^\nu}{\partial x^\mu}
		\end{split}
		\end{equation}	
		将(\ref{pr:delta x})和(\ref{pr:delta p})带入(\ref{pr:delta H})得到
		\begin{equation}\label{pr:delta H'}
		\begin{split}
			0=\delta H&=-\epsilon\left[p_\nu\frac{\partial f^\nu}{\partial x^\mu}\frac{\partial H}{\partial p_\mu}-f^\mu\frac{\partial H}{\partial x^\mu}\right]=-\epsilon\dot{Q}.
		\end{split}
		\end{equation}
		要使(\ref{pr:delta H'})对于任意的无穷小量$\epsilon$都成立,唯有$\dot{Q}=0$,得证.\\
		\end{proof}
	
		\begin{theorem}
		守恒荷$Q$是无穷小坐标变换的生成元.
		\end{theorem}
	
		力学量$A(x^\mu;p_\mu)$的变分为
		\begin{eqnarray*}
		\delta A=\frac{\partial A}{\partial x^\mu}\delta x^\mu+\frac{\partial A}{\partial p_\mu}\delta p_\mu,
		\end{eqnarray*}
		从上面对守恒荷的证明可知,$\delta x^\mu$和$\delta p_\mu$能够借助守恒荷$Q$表述为
		\begin{exercise}
			\begin{equation}\label{eq:generator}
				\begin{split}
					\delta x^\mu&=\epsilon f^\mu=\epsilon\{x^\mu,Q\};\\
					\delta p_\mu&=-\epsilon p_\nu\frac{\partial f^\nu}{\partial x^\mu}=\epsilon\{p_\mu,Q\}.
				\end{split}
			\end{equation}
		\end{exercise}
		
		(\ref{eq:generator})请读者自行证明.因此$\delta A$可以进一步写成
		\begin{eqnarray*}
		\delta A=\epsilon\left(\frac{\partial A}{\partial x^\mu}\{x^\mu,Q\}+\frac{\partial A}{\partial p_\mu}\{p_\mu,Q\}\right).
		\end{eqnarray*}
		在经典情形下,生成元构造尚不明显.但在Poisson括号量子化后,可以非常明显的展示出生成元构造.例如
		\begin{eqnarray*}
		x^\mu\rightarrow x'^\mu&&=x^\mu+\delta x^\mu\\
		&&=x^\mu+\epsilon[x^\mu,Q]\\
		&&=x^\mu-\epsilon Qx^\mu+\epsilon x^\mu Q\\
		&&=\left(1-\epsilon Q\right)x^\mu\left(1+\epsilon Q\right)\\
		&&=e^{-\epsilon Q}x^\mu e^{\epsilon Q},
		\end{eqnarray*}
		其中第四个等号省略了$O(\epsilon^2)$.