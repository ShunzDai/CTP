\part{附录}\label{Part:Appendix}
	
		\begin{margintable}\vspace{1.4in}\footnotesize
			\begin{tabularx}{\marginparwidth}{|X}
			Section~\ref{sec:Dimensional}. 常见物理量的单位换算关系\\
			Section~\ref{sec:N-sphere}. $n$维球的体积与表面积\\
			%Section~\ref{sec:N-sphere}. $n$维球的体积与表面积\\
			\end{tabularx}
		\end{margintable}

		\section{常见物理量的单位换算关系}\label{sec:Dimensional}
			表\ref{tab:Dimensional}给出了一些常见物理量或物理常数的单位制换算关系.表格中的自然单位制是$c=\hbar=\varepsilon_0=k=1$的自然单位制,其中$c$是真空光速,$\hbar$是约化Planck常数,$\varepsilon_0$是真空介电常数,$k$是Boltzmann常数.
		
			\begin{table}[!htbp]
			\centering
			
			\begin{tabular*}{\hsize}{@{}@{\extracolsep{\fill}}rrr@{}}
				\toprule
				常数							&				国际单位制					&	自然单位制	\\
				\midrule
				真空光速$c$							&	$2.99792458E+08\ m^{1}s^{-1}$			&	$1$							\\
				Planck常数$h$					&	$6.626E-34\ kg^{1}m^{2}s^{-1}$			&	$1$							\\
				引力常数$G$						&	$6.67259E-11\ kg^{-1}m^{2}s^{-1}$		&	$m^{2}$						\\
				Coulomb常数$k_e$				&	$8.988E+09\ A^{-2}kg^{1}m^{3}s^{-4}$	&	$1$							\\
				基本电荷量$e$					&	$1.602E-19\ A^{1}s^{1}$					&	$1$							\\
				真空介电常数$\varepsilon_0$		&	$8.854E+12\ A^{2}kg^{-1}m^{-3}s^{4}$	&	$1$							\\
				真空磁导率$\mu_0$				&	$4\pi E-7\ A^{-2}kg^{1}m^{1}s^{-2}$		&	$1$							\\
				Boltzmann常数$k$				&	$1.38 E-23\ K^{-1}kg^{1}m^{2}s^{-2}$	&	$1$							\\
				\midrule
				力学量							&				国际单位制					&	自然单位制	\\
				\midrule
				长度$L$							&	$m^{1}$									&	 $m^{1}$					\\
				时间$T$							&	$s^{1}$									&	 $m^{1}$					\\
				质量$M$							&	$kg^{1}$								&	 $m^{-1}$					\\
				动量$P$							&	$kg^{1}m^{1}s^{-1}$						&	 $m^{-1}$					\\
				能量$H$							&	$kg^{1}m^{2}s^{-2}$						&	$m^{-1}$					\\
				能量密度						&	$kg^{1}m^{-1}s^{-2}$					&	$m^{-4}$					\\
				能量流密度$S$					&	$kg^{1}s^{-3}$							&	$m^{-4}$					\\
				能动量张量分量$T_{\mu\nu}$		&	$m^{-2}$								&	$m^{-2}$					\\
				Einstein张量分量$G_{\mu\nu}$	&	$m^{-2}$								&	$m^{-2}$					\\
				\midrule
				电磁学量						&				国际单位制					&	自然单位制	\\
				\midrule
				电流$I$							&	$A^{1}$									&	 $m^{-1}$					\\
				电场强度$E$						&	$A^{-1}kg^{1}m^{1}s^{-3}$				&	 $m^{-2}$					\\
				%电位移矢量$D$					&	$m^{1}$									&	 $m^{1}$					\\
				%磁场强度$H$					&	$m^{1}$									&	 $m^{1}$					\\
				磁感应强度$B$					&	$A^{-1}kg^{1}s^{-2}$					&	 $m^{-2}$					\\
				规范场分量$A$					&	$A^{-1}kg^{1}m^{1}s^{-2}$				&	 $m^{-1}$					\\
				场强张量分量$F_{\mu\nu}$		&	$A^{-1}kg^{1}s^{-2}$					&	 $m^{-2}$					\\
				\midrule
				热学量							&				国际单位制					&	自然单位制	\\
				\midrule
				温度$\varTheta$					&	$K^{1}$									&	 $m^{-1}$					\\
				$\cdots$						&	$\cdots$								&	 $\cdots$					\\




					
				\bottomrule
			\end{tabular*}
			\caption{常见物理量的单位换算关系}\label{tab:Dimensional}
			\end{table}
			
			\newpage


		\section{$n$维球的体积与表面积}\label{sec:N-sphere}

			这一节我们研究$n$维球的体积与表面积问题,它有一个优雅的解决方案,即使用Gauss 积分.首先我们明确这样一个性质,即一个嵌入$\mathbb{R}^n$中的、半径为$r$的$n$维球$S^{n-1}$,其体积$V_n(r)$一定正比于$r^n$,比例系数是单位球的体积$V_n(1)$:
			\begin{eqnarray}\label{eq:V_n(r)}
				V_n(r)=V_n(1)r^n. 
			\end{eqnarray}
			我们对(\ref{eq:V_n(r)})微分可以得到表面积满足
			\begin{eqnarray}\label{eq:S_n(r)}
				dV_n(r)=nV_n(1)r^{n-1}dr=S_n(r)dr, 
			\end{eqnarray}
			其中$S_n(r)\equiv =nV_n(1)r^{n-1}$.我们在超极坐标下构造这样一个Gauss型积分
			\begin{eqnarray}\label{eq:Gauss int}
				\int_{\mathbb{R}^n}e^{-r^2}dV_n,
			\end{eqnarray}
			现在我们尝试求出(\ref{eq:Gauss int}).若取坐标系为超球坐标系,则有
			\begin{eqnarray}\label{eq:Gauss int1}
				\int_{\mathbb{R}^n}e^{-r^2}dV_n=nV_n(1)\int_0^{+\infty}r^{n-1}e^{-r^2}dr,
			\end{eqnarray}
			作一个变量替换$u=r^2$,代入得
			\begin{eqnarray*}
				\int_0^{+\infty}r^{n-1}e^{-r^2}dr&=&\frac{1}{2}\int_0^{+\infty}u^{\frac{n}{2}-1}e^{-u}du=\frac{1}{2}\varGamma\left(\frac{n}{2}\right),
			\end{eqnarray*}
			于是
			\begin{eqnarray}\label{eq:Gauss int2}
				\int_{\mathbb{R}^n}e^{-r^2}dV_n(r)=V_n(1)\frac{n}{2}\varGamma\left(\frac{n}{2}\right)=V_n(1)\varGamma\left(\frac{n}{2}+1\right);
			\end{eqnarray}
			若取坐标系为直角坐标系,则有
			\begin{eqnarray}\label{eq:Gauss int3}
				\int_{\mathbb{R}^n}e^{-r^2}dV_n=\int_{\mathbb{R}^n}e^{-\left[(x^1)^2+\cdots+(x^n)^2\right]}dx^1\cdots dx^n=\left(\int_{-\infty}^{+\infty}e^{-x^2}dx\right)^n,
			\end{eqnarray}
			作一个变量替换$u=x^2$,代入得
			\begin{eqnarray*}
				\int_{-\infty}^{+\infty}e^{-x^2}dx=\int_{0}^{+\infty}u^{-\frac{1}{2}}e^{-u}dx=\varGamma\left(\frac{1}{2}\right)=\pi^{\frac{1}{2}},
			\end{eqnarray*}
			于是
			\begin{eqnarray}\label{eq:Gauss int4}
				\int_{\mathbb{R}^n}e^{-r^2}dV_n=\pi^{\frac{n}{2}}.
			\end{eqnarray}
			结合(\ref{eq:V_n(r)}),(\ref{eq:S_n(r)}),(\ref{eq:Gauss int2})以及(\ref{eq:Gauss int4})我们得到
			\begin{eqnarray}
				V_n(r)&=&\frac{\pi^{\frac{n}{2}}}{\varGamma\left(\frac{n}{2}+1\right)}r^n;\\
				S_n(r)&=&\frac{2\pi^{\frac{n}{2}}}{\varGamma\left(\frac{n}{2}\right)}r^{n-1}.
			\end{eqnarray}


			