\part{附录}\label{Part:Appendix}
	
	\section{自述}\label{sec:account}
		\begin{margintable}\vspace{1.4in}\footnotesize
			\begin{tabularx}{\marginparwidth}{|X}
			Section~\ref{sec:account}. 自述\\
			\end{tabularx}
        \end{margintable}
        2016年7月,我最终还是拿到了西华师范大学物理与空间科学学院的录取通知书。在此之前,我一度以为自己只能读专科了,想着中学毕业后到武汉铁路职业技术学院学习开地铁,然后当个地铁司机。唯一令我不解的是,我填写志愿时网站上写的是“物理专业”,而当录取通知下来时却变成了“物理专业(师范)”。我不知道的是,多出的这两个字便是噩梦的开始。但不管怎样,我终于有学上了。那年暑假,我发了条微博,大意是我想奉献我的一生投入物理研究。不久后便有前辈评论说我太年轻太简单太幼稚,大部分搞科研的都是水文章的和搬砖的。但那时的我还不能理解。

	    后来我便入学了。在我的想象中,大学是学术自由的地方,我可以学习任何我想学的东西。2016年9月30日,学院举办了西华师范大学天文系的成立仪式,我见到了几位院士、天文系主任邓李才研究员以及朱进老师。我们学校作为一个二本学校,成为了新中国历史上第7所具有天文系的学校,在当时来看这是一件非常令人骄傲的事。然而刚高兴不久,期末考试立刻就给了我一个下马威,将我的大学梦击得粉碎:我期末正考一下就挂了两科,一门是体育1(公共必修课),一门是书法训练(教师教育必修课),而我的第一门专业必修课,力学,也只有不到70分,高等数学1也不到80分。所幸的是,在2017年初的补考中,我顺利的通过了补考,至少在当时我还没有挂科记录。但按照硬性条件,由于我有了两次补考记录,我大学生涯一上来就失去保研资格。
	
	    大一下学期是一个好的开始,这学期我们学习了电磁学和高等数学2,前者让我初步认识到梯度、散度、旋度的物理意义,而后者的矢量分析部分着实令我兴奋。在这一学期的学习中,我注意到正交曲线坐标系的Jacobian恰好就是几个Lamé系数的乘积。这个发现尤其重要,是这场噩梦中为数不多的让人欣慰的部分。可以说,我大学生涯的大部分时间都是为了找到一个解释:为什么正交曲线坐标系的Jacobian恰好就是几个Lamé系数的乘积?同学期,我在微博上认识了日本东北大学的李韬瀚博士,虽然他嘴臭,开玩笑没下限,但他在大学期间给予了我莫大的帮助,是为数不多的令我感激的人。通过李博士,我短暂的结识了一位复旦物理系的前辈,她比我早一年入学。通过她,我得知复旦的理论力学课程上会把Lamé系数称为“度规系数”,这里出现了“度规”一词。而当我查找与“度规”相关的资料时,搜索结果将我导向了“微分几何”。2017年6月份,我购买了一本《微分几何与拓扑学简明教程》,通过这本书开头几页的内容,我逐渐了解到矢量分析背后庞大的数学体系。这个过程很像是一场解谜游戏:我意外获得了一个上锁的箱子,通过与其他玩家的交流,我得知了钥匙的位置信息,然后我又去寻找钥匙,通过一个人的奋斗以及历史的行程,我找到了钥匙并打开箱子,发现箱子里面装了另一把钥匙,新的目标是寻找与这把新钥匙对应的新箱子……这是一个探索的过程,是与未知对抗的过程,我沉溺于其中。当年7月,闫沐霖教授到我们学院做了一次报告,题目是de Sitter不变的狭义相对论。可惜我当时学识尚浅,没能听懂。7月中旬,在这学期最终的期末考试中,我正考又挂了一科英语2。

	    2017年暑假,我因挂科逐渐变得焦虑,在班级群发出的成绩单中,我是唯一一个专业成绩靠前、正考挂科次数也靠前的学生。在度过焦虑的假期后,我迎来了大二上学期,也迎来了英语2的补考。我的补考终于还是挂了,这意味着我要进行重修了。即便要进行重修,我还是将重心放在了理论力学与线性代数课程上,为此,我购买了《经典力学的数学方法》和全套《代数学引论》。学习理论力学时,我把重心全部放在了分析力学上,把Lagrange力学学得滚瓜烂熟,但对Hamilton力学不求甚解,这是因为最开始我不能很好的理解Legendre变换以及相空间等构造;而学习线性空间、对偶空间和张量的公理化构造时,我一度陷入困境,没有办法时甚至将书上的对应内容抄了五六遍,效果当然微乎其微。大二上学期,我们新增了公共选修课,同学们纷纷选择最容易通过的两性教育,我抢不过,所以选择了物理系某位“三清博士”开设的“物理学前沿讲座”。这门课程中,“三清博士”仅作为主持人的角色,真正讲课的是由他邀请的学院的不同方向的老师。通过这门课,我认识了王旭东老师。王老师是“非交换几何”的专家,他告诉我们,物理学院未来会给本科生提供学习理论物理的机会。我得到了他的联系方式,在2017年11月初通过简单的面试后,开始学习广义相对论与量子场论。说上去高大上,实际上就是允许本科生旁听研究生的课程。与我一同的本科生中,有几个同年级的公费师范生以及一位军训时带我们寝室的师姐。一开始我当然是没听懂的,即便我晓得什么是“度规”,但还是不能理解那无穷无尽的指标以及升降指标的机制。师姐倒是记了很多笔记,我却认为不懂得原理,记了也没用,所以没怎么记笔记。王老师要求旁听的本科生每周都要去办公室问问题,我即便没听懂,还是硬着头皮去请教他,问题无非是不懂指标的含义、升降指标的规则。王老师听了很生气,他说你又不记笔记,怎么能懂?后来我还是没怎么做笔记,但我下狠心买了全套的《微分几何入门与广义相对论》,这套书在11月11日到货。那天我还在参加英语2的重修课程,中午取了快递之后就返回了重修教室。我确认收货时发现双十一活动中这套书打折降价了很多,我联系店家希望能退差价。得知不能退差价后,我很沮丧。我告诉店家当天是我的生日,能不能祝我生日快乐,店家满足了我的要求。改变是在学习协变导数时,仿佛突然间一切都变得顺理成章了:普通导数算符作用到矢量分量上不满足张量变换规律,于是修正普通导数算符得到协变导数算符,它作用到矢量分量上的结果还是协变的;约定一种作用到度规上为零的协变导数算符,这种协变导数算符与普通导数算符的差异体现在联络上;联络可以通过轮换指标,用Christoffel符号表述;测地线满足最小作用量原理,通过推导线元的变分,可以得到测地线方程……到了12月的时候,我已逐步理解Riemann曲率张量以及Einstein场方程的结构了。但对于量子场论,我还是一窍不通。这学期我又挂了英语3,重修的英语2还是没过。此时的我就像一只惊弓之鸟,对一切考试都无比焦虑。在2018年1月回家的火车上,我向我的理论力学老师发了一封邮件,我问他我的学习有没有“好高骛远”的嫌疑。这个问题我不敢问王老师,之前上课时我曾隐晦的向王老师传达过这个观点,王老师有些愠怒的说,本科的知识就像是自行车,骑得再好也比不上汽车。不久后理论力学老师回复我,说这个问题最好面谈。我后来还是没有去找他,不了了之。

	    2018年是我最困难的一年。年初教育部发文,表明要规范高校毕业生质量,那时的官方语言叫做“严格把关”。我现在回想起来,我那时可能已经隐约感到学校要学习这一精神了,这大概在潜意识里对我造成了巨大的压力。3月份,我第一次去找辅导员,我告诉他,我英语太差了,估计我要毕不了业了。辅导员想不通,他说你都毕不了业,那谁还毕得了业,四级过不了还有校四级,挂科过不了还有大补考,言下之意就是毕业要求水得很,我不必多虑。但没过多久,一次班会上辅导员就表示校四级和大补考取消了,这对我的“学术自由”是一记重击。一开始我想休学调整,但后来选择了心理咨询。从6月开始直到期末考试,我一度要进行每周一次的心理咨询。第一次去心理咨询时,我希望获得一份心理状况的评估,我填了一个量表,得到的反馈结果是轻度抑郁,重度焦虑。但没过多久我就发现学校的心理咨询中心也水得很,我其实并不是真的需要心理咨询,我只是想找人说话罢了。那学期我们正在学习数学物理方法和原子物理学,我认为数学物理方法里的内容都是无用的技巧,靠计算机就行了,学生不需要当“人肉计算机”;而原子物理我哪怕到了现在也一窍不通。我还和电工学实验的老师的研究生吵了一架,最后实验考试我都没去。我告诉王老师,我状态太差了,不想做了。王老师说没事。大二上学期的期末考试当然考得稀烂,挂了原子物理、教育学、马克思主义基本原理和教师基本素养,未修到的学分绩点累计达到了18点,再多2点我就要留级了。那时是我平均学分绩点最低的时候,只有1。59点,哪怕是现在\footnote{指的是2019年中旬.},我也只有1。69点,勉强达到了毕业的1。60点要求。

	    2018年暑假,我终于在家人面前崩溃了,我大哭了一场,整个人都是呆滞的:唾液从唇边流下,流成一条线,母亲看到后问我是不是被狗咬过;有一次我前往家附近的超市,准备从后门出去时被店员拦住,店员说要从前门走,她看我的眼神是看精神病的那种眼神。9月,经过了一个暑假的休整,大三上学期开学了。我也逐步恢复过来,补考通过了教师基本素养和教育学,但原子物理和马克思主义基本原理我还是挂了,需要重修。那学期我重修了马克思主义基本原理和英语2,虽然“马原”的成绩录入出了点问题,但两门课程的重修考试我都通过了。我只剩下两门课程没有通过:英语3和原子物理。

	    2019年初,我参与了为期一学期的教育实习,也就是到高中当实习老师。实习期间,我常常会忘记我还在读大学,更多的时候我感到每天的经历都是一样的,每天都在做同样的事,就像动物园里的动物,进行着无时不刻的刻板行为。实习开始时,我说我想让学生高屋建瓴的学习物理,结果学生对物理根本就不感兴趣,没有探索的欲望。或者说,高中不是要培养对某一学科具有较强认知的学生,而是要培养德智体美劳全面发展的学生。不光是高中,哪怕是大学对物理专业的学生的培养方案,也是要求学生德智体美劳全面发展。实习期间有过几个同学来办公室问我相对论是什么,强相互作用力是什么,但这超出了他们的理解范围,甚至后者也超出了我的理解范围,我只能给他们一个虚无缥缈的回答,他们似懂非懂,然后说“谢谢老师”,就走开了。他们总是对高大上的物理问题感兴趣,例如暗物质,弦论等内容,而要理解这些内容,需要花很多功夫去学习基础知识,否则就真的有好高骛远之嫌。对物理学抱有极大兴趣的人是少数,他们是纯粹的理想主义者,活在自己的世界里,理想推动着他们前进;更多的人是无比现实的,生活推动着他们前进。理想主义者需要现实一点,现实主义者也要有一点梦想,如何在这两者间找到一个平衡状态,这才是老师应该教会学生的。我看到有的学生文化课很糟,什么都学不进去,但他们也许充满了正义感,也许在平时乖巧可人,他们也能很快乐。有学生对我说,你为什么要去学这些东西,你看你愁眉苦脸的,你看我多开心。我觉得很有道理,人作为动物,活着才是最重要的,然后才是满足自我实现需求。我想,我或许应该放下点东西了。

	    2019年7月,我在王老师的支持下,前往贵州参加了引力协会组织的相对论学术年会,了解了大部分物理工作者的日常工作。我发现许多前辈的工作根本不涉及微分几何,我甚至没有听到或看到几个张量结构,这令我非常惊讶,也很令人沮丧:一开始,别人说我这是好高骛远,要把基础打好。那时我还会自省,觉得别人说的有道理,本科生毛都没长齐还做什么科研?现在我真的了解到别人所做的工作,发现别人的基础可能还没有我好,我便开始怀疑:到底什么才是“基础好”?Yau等人得到正质量定理,Yau的物理基础好吗?引力会议上作报告的物理工作者不会微分几何,他们的数学基础好吗?于是我不再随便相信他人对我的“建议”。对一个人的“建议”一定要是客观的、符合实际情况的。如果一个人根本不了解我,只是刚刚接触我,了解到我的工作后便不假思索的擅自给予我一个“建议”,那这个“建议”还是客观的吗?

	    我开始重新审视我过去的工作。我的大学生涯充满了别人的傲慢与偏见,到底是我真的做得不够好,还是我不够自信?第一位歧视我出身的人是那位与我短暂接触的复旦学姐,她的言语中时常带有一份对二本学生的不屑。有一天我告诉她群是关于对称性的理论,她却对我的表述嗤之以鼻,从那以后我便没有与她往来过。“物理学前沿讲座”的期末考核是让学生提交一份关于物理学的小论文,我写的是关于度规张量与梯度散度旋度、体元的关联,那位“三清博士”给我评了78分。后来他告诉我,我的那篇论文“仅仅是当时你对度规张量的粗浅认识”“对于你论文里的具体内容我并不是太懂”“没有体现出你太多的自己的个人观点”“物理永远不是数学”。我逐渐发现,你与物理工作者谈物理的时候,他会给你吹数学,张口纤维丛,闭口上同调,还会讲甜甜圈和咖啡杯拓扑同胚;而当你真的与他谈数学的时候,例如告诉他集合$X$的拓扑$\mathcal{X}$是$X$的所有子集构成的集合与空集之并,满足如下两条:
	    \begin{enumerate}
		    \item 有限个$X$的子集的交是拓扑的元素;
		    \item 无限个$X$的子集的并是拓扑的元素。
	    \end{enumerate}
	    他就会恼羞成怒:数学不能算物理……数学!……物理学家的事,能叫数学么?接连便是难懂的话,什么“扩大希尔伯特空间”,什么“重整化”之类,引得众人都哄笑起来,教室内外充满了快活的空气。我认为将数学与物理作为对立的两面在很大程度上是反智的、幼稚可笑的。物理的公理化被列为Hilbert著名演讲的数个问题之一,而根据历史进程,我们有理由相信一切物理都是能够公理化的。但不少物理工作者都认为“物理不是数学”,用这种论断将自己麻痹在一种模糊不清的表述之中,竟对此引以为傲。这样的经历数不胜数,而我对此感到厌烦。
		
		

        朝闻道,夕死可矣。

	    \rightline{2019年8月23日,于三峡大学}
        \rightline{2019年9月5日,修订于凉山州}
        \rightline{2020年5月17日,修订于三峡大学}