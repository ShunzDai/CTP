\part{引言}
    我从2019年9月就开始写这份笔记了,但写到今天也只有寥寥数页。原本的笔记也有这样一个引言部分,主要讲述了我整个大学四年的艰难历程,但现在看来,那些文字只不过是自己感动自己罢了\footnote{我仍然保留了那些文字,放在我的博客中。}。现在,我重写了这份引言,用尽量简短的语言来阐述这份笔记做了些什么工作。
	
    这份笔记受微分几何的影响颇深,微分几何也是我大学四年以来尤其喜爱的数学构造。相比于线性代数,微分几何为代数问题赋予了几何的看法,这使得我们能够从不同方向去看同一事物,像盲人摸象一般拼凑出研究对象的各种性质;相比于微积分(当然,我指的是高等数学中的微积分,而不是数学分析中的微积分),微分几何更注重“基本结构”,它没有繁多的运算技巧,不必将问题拘谨在冗长的计算中。事实上,理论物理与微分几何的关系可以说是“殊途同归”,它们都用到了公理化(或者说是“第一性原理”)的方法。在研究这种问题时,我们总可以从几条基本公理(或基本假设)出发,逐步推理出宏伟的结构。
		
    我时常向朋友举例说明普通物理与理论物理的差异,这一差异正如这两个问题间的差异:第一个问题是“305497855加406804039等于多少”,第二个问题是“1加2为什么等于2加1”。对于第一个问题,我认为稍稍用点心便能计算出结果,是只有难度没有深度的问题;而对于第二个问题,我的朋友往往不能理解,他们通常会笑着问我:“这是什么问题,那你说为什么1加2等于2加1?”,这时候我就会回答:“因为加法满足交换律”。我们在小学数学课上就已经学过“交换律”“结合律”“分配律”之类的东西,当我们在大学听到“1加2为什么等于2加1”这种问题时,按理说我们应当立即给出答案才对,但事实是我们往往答不出来。其原因就是我们从小到大受过的数学训练的目的就不是为了记住这些基本结构,而是为了将数学作为一种“工具”进行应用。所以,我认为第二个问题是既有难度也有深度的问题,真要把这“第一性原理”弄清楚,还不是一个简单的事。
		
    在繁琐冗长的计算训练中,我们逐步忘却了那些基本结构的作用。当然,这些论断不意味着我们在学习理论物理时不需要任何实际计算、应用,只是说,我们在学习理论物理时更应该将注意力放在基本结构上,而简单的计算能够加深理解,这就已经足够了,我们不必将精力放在如同上述第一个问题的复杂计算上。

    这份笔记中,我们将通篇使用Einstein求和约定,即相同的一对上下指标表示求和。我们将始终保持上下指标的差异,这种差异原则上与度量张量升降指标无关,而是一种源自事物本身的“对偶”属性。

    第二部分我们回顾了