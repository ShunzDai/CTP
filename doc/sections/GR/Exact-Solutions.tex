\section{引力场方程的精确解}\label{sec:Exact Solutions}
	\subsection{真空静态球对称时空的引力场方程}
        Einstein引力场方程是一个二阶张量方程,在$c=1$的自然单位制下,其形式为
        \begin{equation}
            R_{\mu\nu}-\frac{1}{2}Rg_{\mu\nu}+\Lambda g_{\mu\nu}=8\pi GT_{\mu\nu}
        \end{equation}
        其中$R_{\mu\nu}$为Ricci张量,$R$为Ricci标量,$g_{\mu\nu}$为时空度规,$\Lambda$为宇宙学常数,$T_{\mu\nu}$为应力-能量张量(Stress-energy tensor),$G$为万有引力常数.以下内容在不另外说明的情况下,使用的均是$c=1$的自然单位制.\\
        若用$g^{\mu\nu}$对$(1)$的两个协变指标缩并,可以得到一个标量方程
        \begin{equation}
            -R+4\Lambda=8\pi GT
        \end{equation}
        其中$T$是$T_{\mu\nu}$的迹.将$(2)$代入$(1)$消去Ricci标量,可以得到引力场方程的一个等效形式
        \begin{equation}
            R_{\mu\nu}-\Lambda g_{\mu\nu}=8\pi G(T_{\mu\nu}-\frac{1}{2}Tg_{\mu\nu})
        \end{equation}
        对于真空,能量-应力张量及其迹均为零.于是显而易见的,此时引力场方程为
        \begin{equation}
            R_{\mu\nu}=\Lambda g_{\mu\nu}
        \end{equation}
        
        
        
        由对称性可知,静态球对称度规形式为
        \begin{equation}
            g_{\mu\nu}=diag\left(-A(r),B(r),r^2,r^2\sin\theta\right)
        \end{equation}
        其中$A$和$B$是两个只与半径$r$相关的函数.假设时空坐标的仿射参量为$\tau$,利用$g_{\mu\nu}$构造如下拉氏量
        \begin{equation}
            L=\frac{1}{2}g_{\mu\nu}\dot{x}^\mu\dot{x}^\nu=-A(r)\dot{t}^2+B(r)\dot{r}^2+r^2\dot{\theta}^2+r^2\sin\theta\dot{\varphi}^2
        \end{equation}
        其中$\dot{x}^\mu\equiv dx/d\tau$.
        将$L$代入Euler-Lagrange方程
        \begin{equation}
            \frac{\partial L}{\partial x^\mu}-\frac{d}{d\tau}\frac{\partial L}{\partial \dot{x}^\mu}=0
        \end{equation}
        可以得到四个方程
        \begin{eqnarray*}
            \ddot{t}+\frac{A'}{A}\dot{t}\dot{r}=0\\
            \ddot{r}+\frac{A'}{2B}\dot{t}^2+\frac{B'}{2B}\dot{r}^2-\frac{r}{B}\dot{\theta}^2-\frac{r\sin^2\theta}{B}\dot{\varphi}^2=0\\
            \ddot{\theta}+\frac{2}{r}\dot{r}\dot{\theta}-\sin\theta \cos\theta\dot{\varphi}^2=0\\
            \ddot{\varphi}+\frac{2\cos\theta}{\sin\theta}\dot{\theta}\dot{\varphi}+\frac{2}{r}\dot{r}\dot{\varphi}=0
        \end{eqnarray*}
        又知测地线方程为
        \begin{equation}
            \ddot{x}^\rho+\Gamma^\rho_{\mu\nu}\dot{x}^\mu\dot{x}^\nu=0
        \end{equation}
        将这四个等式与测地线方程作对比,容易看出,静态球对称时空的非零Christoffel符号分别为
        \begin{equation}
            \begin{split}
                    \Gamma^0_{10}&=\Gamma^0_{01}=\frac{A'}{2A}\\
                    \Gamma^1_{00}&=\frac{A'}{2B};\Gamma^1_{11}=\frac{B'}{2B};\Gamma^1_{22}=-\frac{r}{B};\Gamma^1_{33}=-\frac{r\sin^2\theta}{B}\\
                    \Gamma^2_{12}&=\Gamma^2_{21}=\frac{1}{r};\Gamma^2_{33}=-\sin\theta \cos\theta\\
                    \Gamma^3_{23}&=\Gamma^3_{32}=\frac{\cos\theta}{\sin\theta};\Gamma^3_{13}=\Gamma^3_{31}=\frac{1}{r}
            \end{split}
        \end{equation}
        
        Riemann曲率张量为
        \begin{equation}
            {R_{\mu\rho\nu}}^\sigma=-\partial_\mu\Gamma^\sigma_{\rho\nu}+\partial_\rho\Gamma^\sigma_{\mu\nu}-\Gamma^\lambda_{\nu\rho}\Gamma^\sigma_{\mu\lambda}+\Gamma^\lambda_{\mu\nu}\Gamma^\sigma_{\rho\lambda}
        \end{equation}
        对$\rho\sigma$指标缩并,得到Ricci张量为
        \begin{equation}
            R_{\mu\nu}={R_{\mu\rho\nu}}^\rho=-\partial_\mu\Gamma^\rho_{\rho\nu}+\partial_\rho\Gamma^\rho_{\mu\nu}-\Gamma^\lambda_{\nu\rho}\Gamma^\rho_{\mu\lambda}+\Gamma^\lambda_{\mu\nu}\Gamma^\rho_{\rho\lambda}
        \end{equation}
        缩并Christoffel符号可以化简为
        \begin{equation}
            \Gamma^\rho_{\rho\nu}=\partial_\nu ln\sqrt{|g|}
        \end{equation}
        其中$g$为度规的行列式,可以算得
        \begin{equation}
            ln\sqrt{|g|}=ln(ABr^4\sin^2\theta)^\frac{1}{2}=\frac{1}{2}lnA+\frac{1}{2}lnB+2lnr+ln(\sin\theta)
        \end{equation}
        将其代入Ricci张量中就得到
        \begin{equation}
            \begin{split}
                    R_{\mu\nu}&=-\partial_\mu\partial_\nu ln\sqrt{|g|}+\Gamma^\lambda_{\mu\nu}\partial_\lambda ln\sqrt{|g|}+\partial_\rho\Gamma^\rho_{\mu\nu}-\Gamma^\lambda_{\nu\rho}\Gamma^\rho_{\mu\lambda}\\
                    &=-\partial_\mu\partial_\nu \left[\frac{1}{2}lnA+\frac{1}{2}lnB+2lnr+ln(\sin\theta)\right]\\
                    &+\Gamma^\lambda_{\mu\nu}\partial_\lambda \left[\frac{1}{2}lnA+\frac{1}{2}lnB+2lnr+ln(\sin\theta)\right]\\
                    &+\partial_\rho\Gamma^\rho_{\mu\nu}-\Gamma^\lambda_{\nu\rho}\Gamma^\rho_{\mu\lambda}
            \end{split}
        \end{equation}
        计算可知,Ricci张量的非零元素全部在对角线上
        \begin{equation}
            \begin{split}
                    R_{00}&=\frac{A''}{2B}-\frac{A'B'}{4B^2}-\frac{(A')^2}{4AB}+\frac{A'}{rB}\\
                    R_{11}&=-\frac{A''}{2A}+\frac{A'B'}{4AB}+\frac{(A')^2}{4A^2}+\frac{B'}{rB}\\
                    R_{22}&=1-\frac{1}{B}-\frac{rA'}{2AB}+\frac{rB'}{2B^2}\\
                    R_{33}&=\sin^2\theta-\frac{\sin^2\theta}{B}-\frac{r\sin^2\theta A'}{2AB}+\frac{r\sin^2\theta B'}{2B^2}=\sin^2\theta R_{22}
            \end{split}
        \end{equation}
        综上所述,联立$(4)(5)(15)$,可以得到真空静态球对称时空的相互独立的场方程为
        \begin{eqnarray}
            R_{00}&=&\frac{A''}{2B}-\frac{A'B'}{4B^2}-\frac{(A')^2}{4AB}+\frac{A'}{rB}=-\Lambda A\\
            R_{11}&=&-\frac{A''}{2A}+\frac{A'B'}{4AB}+\frac{(A')^2}{4A^2}+\frac{B'}{rB}=\Lambda B\\
            R_{22}&=&1-\frac{1}{B}-\frac{rA'}{2AB}+\frac{rB'}{2B^2}=\Lambda r^2
        \end{eqnarray}
        其中缺少了$R_{33}$分量的方程,这是因为$R_{33}$分量列出的方程是一个冗余的方程.\\
        
        
        下面分$\Lambda=0$及$\Lambda\ne0$两种情况分别说明场方程解的具体形式.
        \subsubsection{Schwarzschild解}
        $\Lambda=0$对应的场方程为
        \begin{eqnarray}
            R_{00}&=&\frac{A''}{2B}-\frac{A'B'}{4B^2}-\frac{(A')^2}{4AB}+\frac{A'}{rB}=0\\
            R_{11}&=&-\frac{A''}{2A}+\frac{A'B'}{4AB}+\frac{(A')^2}{4A^2}+\frac{B'}{rB}=0\\
            R_{22}&=&1-\frac{1}{B}-\frac{rA'}{2AB}+\frac{rB'}{2B^2}=0
        \end{eqnarray}
        用$B$乘以$(19)$,$A$乘以$(20)$,算得
        \begin{equation}
            BR_{00}+AR_{11}=\frac{A'B+AB'}{B}=\frac{(AB)'}{B}=0
        \end{equation}
        积分得到
        \begin{equation}
            AB=k_1
        \end{equation}
        其中$k_1\in R$是积分常数.将$(23)$代入$(21)$,得到
        \begin{equation}
            k_1=A+rA'=(rA)'
        \end{equation}
        积分得到
        \begin{equation}
            A=k_1+\frac{k_2}{r}
        \end{equation}
        其中$k_2\in R$是积分常数.将$(25)$带回$(23)$得到
        \begin{equation}
            B=\frac{k_1}{k_1+\frac{k_2}{r}}
        \end{equation}
        时空线元即
        \begin{equation}
            ds^2=-\left(k_1+\frac{k_2}{r}\right)dt^2+\frac{k_1}{k_1+\frac{k_2}{r}}dr^2+r^2d\theta^2+r^2\sin^2\theta d\varphi^2
        \end{equation}
        观察$27$可知,这一线元对应的时空在无穷远处应渐近平直
        \begin{equation}
            {\lim_{r \to \infty}A}={\lim_{r \to \infty}B}=1
        \end{equation}
        因此可得积分常数$k_1=1$.另一个积分常数$k_2$可由测地线方程的弱场近似求得,其值为$-2GM$,其中$M$为引力源的质量.这样,便求得度规为
        \begin{equation}
            ds^2=-\left(1-\frac{2GM}{r}\right)dt^2+\frac{1}{1-\frac{2GM}{r}}dr^2+r^2d\theta^2+r^2\sin^2\theta d\varphi^2
        \end{equation}
        这一解是Karl Schwarzschild率先得到的,因此又被称为Schwarzschild解.
        
        
        
        \subsubsection{Schwarzschild-de Sitter解}
        
        
        
        $\Lambda\ne0$对应的场方程就是$(16)(17)(18)$本身.使用Schwarzschild解中同样的方法,用$B$乘以$(16)$,$A$乘以$(17)$,算得
        \begin{equation}
            BR_{00}+AR_{11}=\frac{A'B+AB'}{B}=\frac{(AB)'}{B}=0
        \end{equation}
        积分得到
        \begin{equation}
            AB=k_3
        \end{equation}
        其中$k_3\in R$是积分常数.将$(31)$代入$(18)$,得到
        \begin{equation}
            k_3=A+rA'+\Lambda r^2=(rA)'+\Lambda r^2
        \end{equation}
        积分得到
        \begin{equation}
            A=k_3+\frac{k_4}{r}-\frac{\Lambda}{3}r^2
        \end{equation}
        其中$k_4\in R$是积分常数.将$(33)$带回$(31)$得到
        \begin{equation}
            B=\frac{k_3}{k_3+\frac{k_4}{r}-\frac{\Lambda}{3}r^2}
        \end{equation}
        时空线元即
        \begin{equation}
            ds^2=-\left(k_3+\frac{k_4}{r}-\frac{\Lambda}{3}r^2\right)dt^2+\frac{k_3}{k_3+\frac{k_4}{r}-\frac{\Lambda}{3}r^2}dr^2+r^2d\theta^2+r^2\sin^2\theta d\varphi^2
        \end{equation}
        考虑到$\Lambda=0$时这一线元应退化到Schwarzschild度规,于是有$k_3=k_1=1,k_4=k_2=-2GM$.这样,便求得度规为
        \begin{equation}
            ds^2=-\left(1-\frac{2GM}{r}-\frac{\Lambda}{3}r^2\right)dt^2+\frac{1}{1-\frac{2GM}{r}-\frac{\Lambda}{3}r^2}dr^2+r^2d\theta^2+r^2\sin^2\theta d\varphi^2
        \end{equation}
        这一度规构造的时空称为Schwarzschild-de Sitter时空.