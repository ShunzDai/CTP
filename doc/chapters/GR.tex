\part{广义相对论}\label{Part:GR}
	\section{微分几何}\label{sec:DG}
		\begin{margintable}\vspace{1.4in}\footnotesize
			\begin{tabularx}{\marginparwidth}{|X}
			Section~\ref{sec:DG}. 微分几何\\
			Section~\ref{sec:structure}. 结构数学\\
			Section~\ref{sec:LieAlgebra}. Lie代数\\
			\end{tabularx}
		\end{margintable}
	\subsection{微分流形}
	
	
	\subsection{张量场与微分形式场}
	
	
	\subsection{联络与曲率}
		对于任意标量场$\phi\in T_{(0,0)}(M)$,其外微分是一个张量场
		\begin{equation}
			\begin{split}
				 d\phi=\frac{\partial\phi}{\partial x^\mu}dx^\mu=\frac{\partial\phi}{\partial {x}^\mu}\frac{\partial{x}^\mu}{\partial {x'}^\nu}d{x'}^\nu=\frac{\partial{\phi'}}{\partial {x'}^\nu}d{x'}^\nu.
			\end{split}
		\end{equation}
		这是说,一个自然的问题是:对于切矢量场或余切矢量场,其外微分是否仍然是张量场?我们以余切矢量场为例,计算其外微分在坐标变换下的性质.
		
		对于一个余切矢量场$\omega\in T_{(0,1)}(M)$,借助局部坐标系可将其表述为
		\begin{equation}\label{eq:domega}
			\begin{split}
				\omega=\omega_\mu dx^\mu=\omega_\mu\frac{\partial{x}^\mu}{\partial {x'}^\nu}d{x'}^\nu=\omega'_\nu d{x'}^\nu,
			\end{split}
		\end{equation}
		\ref{eq:domega}的全微分在坐标变换下满足
		\begin{equation}
			\begin{split}	
			d\omega &=d\omega'_\nu{\wedge}d{x'}^\nu\\
			&= d\left(\omega_\mu\frac{\partial{x}^\mu}{{\partial}{x'}^\nu}\right)\wedge d{x'}^\nu\\
			&=\frac{\partial{x}^\mu}{{\partial}{x'}^\nu}d\omega_\mu{\wedge}d{x'}^\nu+\omega_{\mu}d\left(\frac{\partial{x}^\mu}{\partial {x'}^\nu}\right){\wedge}d{x'}^\nu\\
			&=\left(\frac{\partial{\omega}_\mu}{{\partial}{x}^\sigma}\frac{\partial{x}^\sigma}{\partial {x'}^\rho}\frac{\partial{x}^\mu}{{\partial}{x'}^\nu}+{\omega}_\mu\frac{\partial^2{x}^\mu}{\partial {x'}^\rho{\partial}{x'}^\nu}\right)d{x'}^\rho{\wedge}d{x'}^\nu\\
			&=\frac{\partial\omega'_\nu}{{\partial}{x'}^\rho}d{x'}^\rho{\wedge}d{x'}^\nu.
			\end{split}
		\end{equation}
		注意到过程中产生了一个二阶偏微分项,与张量的变换规律对比可知,这一项破坏了张量的变换规律.但外代数的全反对称特性可以消除掉这一项带来的影响,即
		\begin{equation}
			\begin{split}
				\frac{\partial\omega'_\nu}{{\partial}{x'}^\rho}d{x'}^\rho{\wedge}d{x'}^\nu&=2\partial'_{[\rho}\omega'_{\nu]}{dx'}^{\rho}\otimes{dx'}^\nu\\
				&=2\left(\partial_{[\sigma}\omega_{\mu]}\frac{\partial{x}^\sigma}{\partial {x'}^\rho}\frac{\partial{x}^\mu}{{\partial}{x'}^\nu}+{\omega}_\mu\frac{\partial^2{x}^\mu}{\partial {x'}^{[\rho}{\partial}{x'}^{\nu]}}\right){dx'}^{\rho}\otimes{dx'}^\nu\\
			&=2\partial_{[\sigma}\omega_{\mu]}\frac{\partial{x}^\sigma}{\partial {x'}^\rho}\frac{\partial{x}^\mu}{{\partial}{x'}^\nu}{dx'}^{\rho}\otimes{dx'}^\nu,
			\end{split}
		\end{equation}
		所以$d\omega$实际上仍是张量场。但微分形式场说到底只是一种特殊的张量场。对于一般的张量场,我们无法通过张量场自身的全反对称性消除上述二阶偏微分项。并且,我们也没有定义过切矢量场的外微分。这些困难意味着外微分运算不能直接推广到张量空间$T_{(p,q)}(M)$上去,于是,我们希望构造出一种定义在张量空间上的新的微分运算,使得任意张量场在这种微分运算下满足张量的变换规律。为此,我们引入\textbf{联络}。
	
		\subsubsection{仿射联络}
		现代微分几何中的联络指的是一般矢量丛上的联络.这一节我们先讨论切丛$T_{(1,0)}(M)$上的联络(称为\textbf{仿射联络}),然后逐步构造出张量丛$T_{(p,q)}(M)$上的联络.下面,我们直接给出仿射联络的定义.
		\begin{definition}
			仿射联络是一个映射
			\begin{equation*}
				D:T_{(1,0)}(M)\rightarrow T_{(1,1)}(M),
			\end{equation*}
			它满足下列条件:
			\begin{enumerate}
				\item 对任意的$A,B\in T_{(1,0)}(M)$有 D(A+B)=D(A)+D(B);						
				\item 对任意的$A\in T_{(1,0)}(M),\alpha\in T_{(0,0)}(M)$有$D(\alpha A)=d\alpha\otimes A+\alpha D(A)$.
			\end{enumerate}
		\end{definition}
		局部上,联络由一组1微分形式给出.我们先看自然标架场$\{\partial_\mu\}$的联络.命
		\begin{equation}
		D(\partial_\mu)=\omega^\rho_\mu \otimes\partial_\rho={\Gamma^\rho}_{\mu\nu}dx^\nu\otimes \partial_\rho,
		\end{equation} 
		其中$\omega^\rho_\mu={\Gamma^\rho}_{\mu\nu}dx^\nu$,${\Gamma^\rho}_{\mu\nu}$称为\textbf{联络系数},它是局部坐标系中的光滑函数.以$\omega^\rho_\mu$作为矩阵$\omega$第$\rho$行第$\mu$列的元素,这样构造的矩阵$\omega$称为\textbf{联络方阵}.可见,任意两个联络间的差异完全体现在联络方阵上.现在,我们来看联络的变换规律.在标架变换下,由联络的定义立即得到			 
		\begin{equation}
			\begin{split}
			D(\partial'_\nu) &= D\left(\frac{\partial{x^\mu}}{\partial{x'}^\nu}{\partial}_\mu\right) \\
			&= d\left(\frac{\partial{x^\mu}}{\partial{x'}^\nu}\right)\otimes{\partial}_\mu+\frac{\partial{x^\mu}}{\partial{x'}^\nu}D\left({\partial}_\mu\right)\\
			&= \left[d\left(\frac{\partial{x^\rho}}{\partial{x'}^\nu}\right)+\frac{\partial{x^\mu}}{\partial{x'}^\nu}\omega^\rho_\mu\right]\otimes{\partial}_\rho\\
			&= \left[d\left(\frac{\partial{x^\rho}}{\partial{x'}^\nu}\right)\frac{\partial{{x'}^\sigma}}{\partial{x^\rho}}+\frac{\partial{x^\mu}}{\partial{x'}^\nu}\omega^\rho_\mu\frac{\partial{{x'}^\sigma}}{\partial{x^\rho}}\right]\otimes\partial'_\sigma\\
			&={\omega'}^\sigma_\nu\otimes\partial'_\sigma,
			\end{split}
		\end{equation}
		其中
		\begin{equation}\label{eq:connection}
		{\omega'}^\sigma_\nu=d\left(\frac{\partial{x^\rho}}{\partial{x'}^\nu}\right)\frac{\partial{{x'}^\sigma}}{\partial{x^\rho}}+\frac{\partial{x^\mu}}{\partial{x'}^\nu}\omega^\rho_\mu\frac{\partial{{x'}^\sigma}}{\partial{x^\rho}},
		\end{equation}
		这就是联络方阵在局部标架场改变时的变换公式.我们还可以进一步展开\ref{eq:connection},得到
		\begin{equation}
		{{\Gamma'}^\sigma}_{\nu\lambda}=\frac{\partial^2{x^\rho}}{\partial{x'}^\lambda\partial{x'}^\nu}\frac{\partial{{x'}^\sigma}}{\partial{x^\rho}}+{\Gamma^\rho}_{\mu\alpha}\frac{\partial{x^\mu}}{\partial{x'}^\nu}\frac{\partial{x}^\alpha}{\partial{x'}^\lambda}\frac{\partial{{x'}^\sigma}}{\partial{x^\rho}},
		\end{equation}
		这正是通常意义下的联络变换公式.可见联络系数${\Gamma^\rho}_{\mu\nu}$并不是一个张量,但引入它能使$D(\partial_\mu)$遵循张量的变换规律.但稍作思考可以发现,若使用全反对称化的技巧,我们也能利用联络系数构造出一个张量.我们将在下一节给出具体讨论.
	
		切丛的联络在余切丛上诱导出一个联络(仍记为$D$),易看出它是从$T_{(0,1)}(M)$到$T_{(0,2)}(M)$的映射.
		\begin{remark}
			相比于外微分算子,可见余切丛上的诱导联络算子将值域从外形式丛的截面(元素即全反对称的张量场)扩展到了一般的张量丛的截面(元素即一般的张量场).所以我们说联络是外微分算子在一般张量丛上的推广.
		\end{remark}
		现在现在我们来考察$\{\partial_\mu\}$的对偶标架$\{dx^\mu\}$的联络.$D(dx^\mu)$由下式确定
		$$d\left(\partial_\mu,dx^\nu\right)=\left(D(\partial_\mu),dx^\nu\right)+\left(\partial_\mu,D(dx^\nu)\right),$$
		我们设
		$$D(dx^\nu)={\omega^*}^\nu_\rho\otimes{dx}^\rho,$$
		于是
		$$\left(\partial_\mu,D(dx^\nu)\right)=\left(\partial_\mu,{\omega^*}^\nu_\rho\otimes{dx}^\rho\right)={\omega^*}^\nu_\rho\delta^\rho_\mu={\omega^*}^\nu_\mu.$$
		考虑到对偶标架满足
		$$\left(\partial_\mu,dx^\nu\right)=\delta^\nu_\mu,$$ 
		于是得到								 
		\begin{equation}
			\begin{split}
				{\omega^*}^\nu_\mu&=\left(\partial_\mu,D(dx^\nu)\right)=-\left(D(\partial_\mu),dx^\nu\right)\\
				&=-\left(\omega^\rho_\mu\otimes\partial_\rho,dx^\nu\right)=-{\omega}^\rho_\mu\delta^\nu_\rho=-{\omega}^\nu_\mu,
			\end{split}
		\end{equation}								 
		即
		$D(dx^\nu)=-{\omega}^\nu_\rho\otimes{dx}^\rho$.
		至此,我们就能计算任意$A\in T_{(1,0)}(M)$与$B\in T_{(0,1)}(M)$的联络了.借助局部坐标系,我们有:								 
		\begin{equation}
			\begin{split}
				D(A)&=D(A^\mu\partial_\mu)=dA^\mu\otimes\partial_\mu+A^{\mu}D(\partial_\mu)\\
				&=(\partial_{\rho}A^\mu+A^{\nu}{\Gamma^\mu}_{\nu\rho})dx^\rho{\otimes}\partial_\mu=\nabla_{\rho}A^{\mu}dx^\rho{\otimes}\partial_\mu;\\
				D(B)&=D(B_{\mu}{dx}^\mu)=dB_\mu\otimes{dx}^\mu+B_{\mu}D({dx}^\mu)\\
				&=(\partial_{\rho}B_\mu-B_\nu{\Gamma^\nu}_{\mu\rho})dx^\rho{\otimes}dx^\mu=\nabla_{\rho}B_{\mu}dx^\rho{\otimes}dx^\mu,
			\end{split}	
		\end{equation}
		其中									 
		\begin{equation}
			\begin{split}
				\nabla_{\rho}A^\mu&=\partial_{\rho}A^\mu+A^{\nu}{\Gamma^\mu}_{\nu\rho};\\
				\nabla_{\rho}B_{\mu}&=\partial_{\rho}B_\mu-B_\nu{\Gamma^\nu}_{\mu\rho}.
			\end{split}		
		\end{equation}								
		这正是通常意义下的\textbf{协变导数}运算.由此可见,协变导数是一种依赖于局部坐标系的分量表述.用类似的方法,我们能得到$T_{(p,q)}(M)$上的诱导联络.
		\begin{definition}
			$T_{(p,q)}(M)$
			设任意$S\in T_{(p,q)}(M)$,借助局部坐标系$\{x^\mu\}$,可将$S$的联络表述为
			\begin{equation}
				\begin{split}
					D(S)&=D({S^{\mu_1\dots\mu_p}}_{\nu_1\dots\nu_q}\partial_{\mu_1}\otimes\dots\otimes\partial_{\mu_p}\otimes{dx}^{\nu_1}\otimes\dots\otimes{dx}^{\nu_q})\\
					&=d({S^{\mu_1\dots\mu_p}}_{\nu_1\dots\nu_q})\otimes\partial_{\mu_1}\otimes\dots\otimes\partial_{\mu_p}\otimes{dx}^{\nu_1}\otimes\dots\otimes{dx}^{\nu_q}\\
					&+{S^{\mu_1\dots\mu_p}}_{\nu_1\dots\nu_q}D(\partial_{\mu_1})\otimes\dots\otimes\partial_{\mu_p}\otimes{dx}^{\nu_1}\otimes\dots\otimes{dx}^{\nu_q}\\
					&+\dots+{S^{\mu_1\dots\mu_p}}_{\nu_1\dots\nu_q}\partial_{\mu_1}\otimes\dots\otimes D({\partial_{\mu_p}})\otimes{dx}^{\nu_1}\otimes\dots\otimes{dx}^{\nu_q}\\
					&+{S^{\mu_1\dots\mu_p}}_{\nu_1\dots\nu_q}\partial_{\mu_1}\otimes\dots\otimes{\partial_{\mu_p}}\otimes D({dx}^{\nu_1})\otimes\dots\otimes{dx}^{\nu_q}\\
					&+\dots+{S^{\mu_1\dots\mu_p}}_{\nu_1\dots\nu_q}\partial_{\mu_1}\otimes\dots\otimes{\partial_{\mu_p}}\otimes{dx}^{\nu_1}\otimes\dots\otimes D({dx}^{\nu_q})\\
					&=(\partial_\rho{S^{\mu_1\dots\mu_p}}_{\nu_1\dots\nu_q}+{S^{\sigma\dots\mu_p}}_{\nu_1\dots\nu_q}{\Gamma^{\mu_1}}_{\sigma\rho}\\
					&+\dots+{S^{\mu_1\dots\sigma}}_{\nu_1\dots\nu_q}{\Gamma^{\mu_p}}_{\sigma\rho}-{S^{\mu_1\dots\mu_p}}_{\sigma\dots\nu_q}{\Gamma^{\sigma}}_{\nu_1\rho}\\
					&-\dots-{S^{\mu_1\dots\mu_p}}_{\nu_1\dots\sigma}{\Gamma^{\sigma}}_{\nu_q\rho})dx^\rho\otimes\partial_{\mu_1}\otimes\dots\otimes\partial_{\mu_p}\otimes{dx}^{\nu_1}\otimes\dots\otimes{dx}^{\nu_q}\\
					&=\nabla_{\rho}{S^{\mu_1\dots\mu_p}}_{\nu_1\dots\nu_q}dx^\rho\otimes\partial_{\mu_1}\otimes\dots\otimes\partial_{\mu_p}\otimes{dx}^{\nu_1}\otimes\dots\otimes{dx}^{\nu_q},
				\end{split}		
			\end{equation}
			其中
			\begin{equation}
				\begin{split}
					\nabla_{\rho}{S^{\mu_1\dots\mu_p}}_{\nu_1\dots\nu_q}&=\partial_\rho{S^{\mu_1\dots\mu_p}}_{\nu_1\dots\nu_q}+{S^{\sigma\dots\mu_p}}_{\nu_1\dots\nu_q}{\Gamma^{\mu_1}}_{\sigma\rho}\\
					&+\dots+{S^{\mu_1\dots\sigma}}_{\nu_1\dots\nu_q}{\Gamma^{\mu_p}}_{\sigma\rho}-{S^{\mu_1\dots\mu_p}}_{\sigma\dots\nu_q}{\Gamma^{\sigma}}_{\nu_1\rho}\\
					&-\dots-{S^{\mu_1\dots\mu_p}}_{\nu_1\dots\sigma}{\Gamma^{\sigma}}_{\nu_q\rho}.
				\end{split}		
			\end{equation}
		\end{definition}
	
	\newpage