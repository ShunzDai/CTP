\part{广义相对论}\label{Part:GR}
	\begin{margintable}\vspace{1.4in}\footnotesize
		\begin{tabularx}{\marginparwidth}{|X}
		Section~\ref{sec:DG}. 微分几何\\
		Section~\ref{sec:structure}. 结构数学\\
		Section~\ref{sec:LieAlgebra}. Lie代数\\
		\end{tabularx}
	\end{margintable}
	\section{微分几何}\label{sec:DG}
		
	\subsection{微分流形}\label{sec:manifold}
	
	
	\subsection{张量场与微分形式场}
	
	
	\subsection{联络与曲率}
		对于任意标量场$\phi\in \varGamma(T_{(0,0)}M)$,其外微分$d\phi$仍是是一个张量场
		\begin{equation}
				 d\phi=\frac{\partial\phi}{\partial x^\mu}dx^\mu=\frac{\partial\phi}{\partial {x}^\mu}\frac{\partial{x}^\mu}{\partial {x'}^\nu}d{x'}^\nu=\frac{\partial{\phi'}}{\partial {x'}^\nu}d{x'}^\nu\in \varGamma(T_{(0,1)}M),
		\end{equation}
		这意味着$d\phi$仍是张量丛截面的元素,用学物理家的话来说也就是$d\phi$仍然满足张量的变换规律.
		
		一个自然的问题是,对于切矢量场或余切矢量场,其外微分是否仍然是张量丛截面的元素?但问题就出在这,外微分算子是一个定义在外形式丛截面上的线性算子,切矢量场不是外形式丛截面中的元素,因此我们无法对一个切矢量场求外微分;余切矢量场作为1微分形式场,虽然它是外形式丛截面的元素,但它只是一种全反对称的张量场,我们无法对一个一般的非全反对称张量场求外微分.这些困难意味着我们需要推广外微分算子,寻找一个定义在一般张量丛截面$\varGamma(T_{(p,q)}M)$上的线性微分算子,让我们能对那些不在外形式丛截面中的张量场进行“微分”运算.这样构造的线性微分算子就是\textbf{仿射联络}.
		\begin{remark}
		余切矢量场$\omega\in \varGamma(T_{(0,1)}M)$在局部坐标系改变时满足
		\begin{equation}\label{eq:domega1}
				\omega=\omega_\mu dx^\mu=\omega_\mu\frac{\partial{x}^\mu}{\partial {x'}^\nu}d{x'}^\nu=\omega'_\nu d{x'}^\nu,
		\end{equation}
		我们对(\ref{eq:domega1})求全微分得到
		\begin{eqnarray}\label{eq:domega2}
			d\omega &=&d\omega'_\nu{\wedge}d{x'}^\nu\notag\\
			&=&d\left(\omega_\mu\frac{\partial{x}^\mu}{{\partial}{x'}^\nu}\right)\wedge d{x'}^\nu\notag\\
			&=&\frac{\partial{x}^\mu}{{\partial}{x'}^\nu}d\omega_\mu{\wedge}d{x'}^\nu+\omega_{\mu}d\left(\frac{\partial{x}^\mu}{\partial {x'}^\nu}\right){\wedge}d{x'}^\nu\notag\\
			&=&\left(\frac{\partial{\omega}_\mu}{{\partial}{x}^\sigma}\frac{\partial{x}^\sigma}{\partial {x'}^\rho}\frac{\partial{x}^\mu}{{\partial}{x'}^\nu}+{\omega}_\mu\frac{\partial^2{x}^\mu}{\partial {x'}^\rho{\partial}{x'}^\nu}\right)d{x'}^\rho{\wedge}d{x'}^\nu\\
			&=&\frac{\partial\omega'_\nu}{{\partial}{x'}^\rho}d{x'}^\rho{\wedge}d{x'}^\nu\notag,
		\end{eqnarray}
		注意到\ref{eq:domega2}中产生了一个二阶偏微分项,与张量的变换规律对比可知,这一项破坏了张量的变换规律.但外微分形式场的全反对称特性可以消除掉这一项带来的影响,即
		\begin{eqnarray}
			\frac{\partial\omega'_\nu}{{\partial}{x'}^\rho}d{x'}^\rho{\wedge}d{x'}^\nu&=&2\partial'_{[\rho}\omega'_{\nu]}{dx'}^{\rho}\otimes{dx'}^\nu\notag\\
				&=&2\left(\partial_{[\sigma}\omega_{\mu]}\frac{\partial{x}^\sigma}{\partial {x'}^\rho}\frac{\partial{x}^\mu}{{\partial}{x'}^\nu}+{\omega}_\mu\frac{\partial^2{x}^\mu}{\partial {x'}^{[\rho}{\partial}{x'}^{\nu]}}\right){dx'}^{\rho}\otimes{dx'}^\nu\notag\\
			&=&2\partial_{[\sigma}\omega_{\mu]}\frac{\partial{x}^\sigma}{\partial {x'}^\rho}\frac{\partial{x}^\mu}{{\partial}{x'}^\nu}{dx'}^{\rho}\otimes{dx'}^\nu,
		\end{eqnarray}
		所以$d\omega$实际上仍是张量场.但微分形式场说到底只是一种特殊的全反对称张量场,对于一般的张量场,我们无法通过张量场自身的对称性消除上述二阶偏微分项.
		\end{remark}

		现代微分几何中的联络指的是一般矢量丛上的联络.这一节我们先讨论切丛$T_{(1,0)}(M)$上的联络(称为\textbf{仿射联络}),然后逐步构造出张量丛$T_{(p,q)}(M)$上的联络.下面,我们直接给出仿射联络的定义.
		\begin{definition}
			仿射联络是一个映射
			\begin{equation*}
				D:\varGamma(T_{(1,0)}M)\rightarrow \varGamma(T_{(1,1)}M),
			\end{equation*}
			它满足下列条件:
			\begin{enumerate}
				\item 对任意的$A,B\in \varGamma(T_{(1,0)}M)$有 D(A+B)=D(A)+D(B);						
				\item 对任意的$A\in \varGamma(T_{(1,0)}M),\alpha\in \varGamma(T_{(0,0)}M)$有$D(\alpha A)=d\alpha\otimes A+\alpha D(A)$.
			\end{enumerate}
		\end{definition}
		局部上,联络由一组1微分形式给出.我们先来看自然标架场$\{\partial_\mu\}$的联络,命
		\begin{equation}
		D(\partial_\mu)=\omega^\rho_\mu \otimes\partial_\rho={\varGamma^\rho}_{\mu\nu}dx^\nu\otimes \partial_\rho,
		\end{equation} 
		其中$\omega^\rho_\mu={\varGamma^\rho}_{\mu\nu}dx^\nu$,${\varGamma^\rho}_{\mu\nu}$称为\textbf{联络系数},它是局部坐标系中的光滑函数.将$\omega^\rho_\mu$作为矩阵$\omega$第$\rho$行第$\mu$列的元素,这样构造的矩阵$\omega$称为\textbf{联络方阵}.可见,任意两个联络间的差异完全体现在联络方阵上.现在,我们来看联络的变换规律.在标架变换下,由联络的定义立即得到
		\begin{eqnarray}
			D(\partial'_\nu)&=&D\left(\frac{\partial{x^\mu}}{\partial{x'}^\nu}{\partial}_\mu\right)\notag\\
			&=&d\left(\frac{\partial{x^\mu}}{\partial{x'}^\nu}\right)\otimes{\partial}_\mu+\frac{\partial{x^\mu}}{\partial{x'}^\nu}D\left({\partial}_\mu\right)\notag\\
			&=&\left[d\left(\frac{\partial{x^\rho}}{\partial{x'}^\nu}\right)+\frac{\partial{x^\mu}}{\partial{x'}^\nu}\omega^\rho_\mu\right]\otimes{\partial}_\rho\notag\\
			&=&\left[d\left(\frac{\partial{x^\rho}}{\partial{x'}^\nu}\right)\frac{\partial{{x'}^\sigma}}{\partial{x^\rho}}+\frac{\partial{x^\mu}}{\partial{x'}^\nu}\omega^\rho_\mu\frac{\partial{{x'}^\sigma}}{\partial{x^\rho}}\right]\otimes\partial'_\sigma\notag\\
			&=&{\omega'}^\sigma_\nu\otimes\partial'_\sigma,
		\end{eqnarray}
		其中
		\begin{equation}\label{eq:connection}
		{\omega'}^\sigma_\nu=d\left(\frac{\partial{x^\rho}}{\partial{x'}^\nu}\right)\frac{\partial{{x'}^\sigma}}{\partial{x^\rho}}+\frac{\partial{x^\mu}}{\partial{x'}^\nu}\omega^\rho_\mu\frac{\partial{{x'}^\sigma}}{\partial{x^\rho}},
		\end{equation}
		这就是联络方阵在局部标架场改变时的变换公式.我们还可以进一步展开(\ref{eq:connection}),得到
		\begin{equation}
		{{\varGamma'}^\sigma}_{\nu\lambda}=\frac{\partial^2{x^\rho}}{\partial{x'}^\lambda\partial{x'}^\nu}\frac{\partial{{x'}^\sigma}}{\partial{x^\rho}}+{\varGamma^\rho}_{\mu\alpha}\frac{\partial{x^\mu}}{\partial{x'}^\nu}\frac{\partial{x}^\alpha}{\partial{x'}^\lambda}\frac{\partial{{x'}^\sigma}}{\partial{x^\rho}},
		\end{equation}
		这正是通常意义下的联络变换公式.可见联络系数${\varGamma^\rho}_{\mu\nu}$并不是一个张量,但引入它能使$D(\partial_\mu)$遵循张量的变换规律.
		
		稍作思考可以发现,若使用全反对称化的技巧,我们也能利用联络系数构造出一个张量.我们将在稍后给出具体讨论.

		切丛截面的联络在余切丛截面上诱导出一个联络(仍记为$D$),易看出它是从$\varGamma(T_{(0,1)}M)$到\\$\varGamma(T_{(0,2)}M)$的映射.
		现在现在我们来考察$\{\partial_\mu\}$的对偶标架$\{dx^\mu\}$的联络.$D(dx^\mu)$由下式确定
		$$d\left(\partial_\mu,dx^\nu\right)=\left(D(\partial_\mu),dx^\nu\right)+\left(\partial_\mu,D(dx^\nu)\right),$$
		我们设
		$$D(dx^\nu)={\omega^*}^\nu_\rho\otimes{dx}^\rho,$$
		于是
		$$\left(\partial_\mu,D(dx^\nu)\right)=\left(\partial_\mu,{\omega^*}^\nu_\rho\otimes{dx}^\rho\right)={\omega^*}^\nu_\rho\delta^\rho_\mu={\omega^*}^\nu_\mu.$$
		考虑到对偶标架满足
		$$\left(\partial_\mu,dx^\nu\right)=\delta^\nu_\mu,$$ 
		于是得到
		\begin{eqnarray}
			{\omega^*}^\nu_\mu&=&\left(\partial_\mu,D(dx^\nu)\right)=-\left(D(\partial_\mu),dx^\nu\right)\\
				&=&-\left(\omega^\rho_\mu\otimes\partial_\rho,dx^\nu\right)=-{\omega}^\rho_\mu\delta^\nu_\rho=-{\omega}^\nu_\mu,
		\end{eqnarray}
		即
		$D(dx^\nu)=-{\omega}^\nu_\rho\otimes{dx}^\rho$.
		至此,我们就能计算任意$A\in T_{(1,0)}(M)$与$B\in T_{(0,1)}(M)$的联络了.在局部坐标系下,我们有:			\begin{eqnarray}
			\begin{split}
				D(A)&=D(A^\mu\partial_\mu)\\
				&=dA^\mu\otimes\partial_\mu+A^{\mu}D(\partial_\mu)\\
				&=(\partial_{\rho}A^\mu+A^{\nu}{\varGamma^\mu}_{\nu\rho})dx^\rho{\otimes}\partial_\mu\\
				&=\nabla_{\rho}A^{\mu}dx^\rho{\otimes}\partial_\mu;
			\end{split}
		\end{eqnarray}
		\begin{eqnarray}
			\begin{split}
				D(B)&=D(B_{\mu}{dx}^\mu)\\
				&=dB_\mu\otimes{dx}^\mu+B_{\mu}D({dx}^\mu)\\
				&=(\partial_{\rho}B_\mu-B_\nu{\varGamma^\nu}_{\mu\rho})dx^\rho{\otimes}dx^\mu\\
				&=\nabla_{\rho}B_{\mu}dx^\rho{\otimes}dx^\mu,
			\end{split}
		\end{eqnarray}
		其中
		\begin{eqnarray}
			\nabla_{\rho}A^\mu&=&\partial_{\rho}A^\mu+A^{\nu}{\varGamma^\mu}_{\nu\rho};\\
				\nabla_{\rho}B_{\mu}&=&\partial_{\rho}B_\mu-B_\nu{\varGamma^\nu}_{\mu\rho}.
		\end{eqnarray}							
		这正是通常意义下的\textbf{协变导数}运算.由此可见,协变导数是一种依赖于局部坐标系的分量表述.用类似的方法,我们能得到$T_{(p,q)}(M)$上的诱导联络.
		\begin{definition}在一般张量丛$T_{(p,q)}M$上,由仿射联络诱导的联络是一个映射
			$$D:\varGamma(T_{(p,q)}M)\rightarrow\varGamma(T_{(p,q+1)}M),$$
			设任意$S\in \varGamma(T_{(p,q)}M)$,则$S$的联络在局部坐标系下满足
			\begin{equation}
				\begin{split}
					D(S)&=D({S^{\mu_1\dots\mu_p}}_{\nu_1\dots\nu_q}\partial_{\mu_1}\otimes\dots\otimes\partial_{\mu_p}\otimes{dx}^{\nu_1}\otimes\dots\otimes{dx}^{\nu_q})\\
				%&=d({S^{\mu_1\dots\mu_p}}_{\nu_1\dots\nu_q})\otimes\partial_{\mu_1}\otimes\dots\otimes\partial_{\mu_p}\otimes{dx}^{\nu_1}\otimes\dots\otimes{dx}^{\nu_q}\notag\\
				%&+{S^{\mu_1\dots\mu_p}}_{\nu_1\dots\nu_q}D(\partial_{\mu_1})\otimes\dots\otimes\partial_{\mu_p}\otimes{dx}^{\nu_1}\otimes\dots\otimes{dx}^{\nu_q}\notag\\
				%&+\dots+{S^{\mu_1\dots\mu_p}}_{\nu_1\dots\nu_q}\partial_{\mu_1}\otimes\dots\otimes D({\partial_{\mu_p}})\otimes{dx}^{\nu_1}\otimes\dots\otimes{dx}^{\nu_q}\notag\\
				%&+{S^{\mu_1\dots\mu_p}}_{\nu_1\dots\nu_q}\partial_{\mu_1}\otimes\dots\otimes{\partial_{\mu_p}}\otimes D({dx}^{\nu_1})\otimes\dots\otimes{dx}^{\nu_q}\notag\\
				%&+\dots+{S^{\mu_1\dots\mu_p}}_{\nu_1\dots\nu_q}\partial_{\mu_1}\otimes\dots\otimes{\partial_{\mu_p}}\otimes{dx}^{\nu_1}\otimes\dots\otimes D({dx}^{\nu_q})\notag\\
				%&=(\partial_\rho{S^{\mu_1\dots\mu_p}}_{\nu_1\dots\nu_q}+{S^{\sigma\dots\mu_p}}_{\nu_1\dots\nu_q}{\varGamma^{\mu_1}}_{\sigma\rho}\notag\\
				%&+\dots+{S^{\mu_1\dots\sigma}}_{\nu_1\dots\nu_q}{\varGamma^{\mu_p}}_{\sigma\rho}-{S^{\mu_1\dots\mu_p}}_{\sigma\dots\nu_q}{\varGamma^{\sigma}}_{\nu_1\rho}\notag\\
				%&-\dots-{S^{\mu_1\dots\mu_p}}_{\nu_1\dots\sigma}{\varGamma^{\sigma}}_{\nu_q\rho})dx^\rho\otimes\partial_{\mu_1}\otimes\dots\otimes\partial_{\mu_p}\otimes{dx}^{\nu_1}\otimes\dots\otimes{dx}^{\nu_q}\notag\\
				&=\nabla_{\rho}{S^{\mu_1\dots\mu_p}}_{\nu_1\dots\nu_q}dx^\rho\otimes\partial_{\mu_1}\otimes\dots\otimes\partial_{\mu_p}\otimes{dx}^{\nu_1}\otimes\dots\otimes{dx}^{\nu_q},
				\end{split}
			\end{equation}
			其中
			\begin{equation}
				\begin{split}
					\nabla_{\rho}{S^{\mu_1\dots\mu_p}}_{\nu_1\dots\nu_q}&=\partial_\rho{S^{\mu_1\dots\mu_p}}_{\nu_1\dots\nu_q}\\
					&+{S^{\sigma\dots\mu_p}}_{\nu_1\dots\nu_q}{\varGamma^{\mu_1}}_{\sigma\rho}+\dots+{S^{\mu_1\dots\sigma}}_{\nu_1\dots\nu_q}{\varGamma^{\mu_p}}_{\sigma\rho}\\
					&-{S^{\mu_1\dots\mu_p}}_{\sigma\dots\nu_q}{\varGamma^{\sigma}}_{\nu_1\rho}-\dots-{S^{\mu_1\dots\mu_p}}_{\nu_1\dots\sigma}{\varGamma^{\sigma}}_{\nu_q\rho}.
				\end{split}		
			\end{equation}
		\end{definition}
		
	
	\newpage