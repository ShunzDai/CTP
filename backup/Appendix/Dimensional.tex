\section{常见物理量的单位换算关系}\label{sec:Dimensional}
	表\ref{tab:Dimensional}给出了一些常见物理量或物理常数的单位制换算关系.表格中的自然单位制是$c=\hbar=\varepsilon_0=k=1$的自然单位制,其中$c$是真空光速,$\hbar$是约化Planck常数,$\varepsilon_0$是真空介电常数,$k$是Boltzmann常数.
	
	\begin{table}[!htbp]
		\centering
		\begin{tabular*}{\hsize}{@{}@{\extracolsep{\fill}}rrr@{}}
			\toprule
			常数							&				国际单位制					&	自然单位制	\\
			\midrule
			真空光速$c$						&	$2.99792458E+08\ m^{1}s^{-1}$			&	$1$							\\
			Planck常数$h$					&	$6.626E-34\ kg^{1}m^{2}s^{-1}$			&	$1$							\\
			引力常数$G$						&	$6.67259E-11\ kg^{-1}m^{3}s^{-2}$		&	$m^{2}$						\\
			Coulomb常数$k_e$				&	$8.988E+09\ A^{-2}kg^{1}m^{3}s^{-4}$	&	$1$							\\
			基本电荷量$e$					&	$1.602E-19\ A^{1}s^{1}$					&	$1$							\\
			真空介电常数$\varepsilon_0$		&	$8.854E+12\ A^{2}kg^{-1}m^{-3}s^{4}$	&	$1$							\\
			真空磁导率$\mu_0$				&	$4\pi E-7\ A^{-2}kg^{1}m^{1}s^{-2}$		&	$1$							\\
			Boltzmann常数$k$				&	$1.38 E-23\ K^{-1}kg^{1}m^{2}s^{-2}$	&	$1$							\\
			\midrule
			力学量							&				国际单位制					&	自然单位制	\\
			\midrule
			长度$L$							&	$m^{1}$									&	 $m^{1}$					\\
			时间$T$							&	$s^{1}$									&	 $m^{1}$					\\
			质量$M$							&	$kg^{1}$								&	 $m^{-1}$					\\
			动量$P$							&	$kg^{1}m^{1}s^{-1}$						&	 $m^{-1}$					\\
			能量$H$							&	$kg^{1}m^{2}s^{-2}$						&	$m^{-1}$					\\
			能量密度						&	$kg^{1}m^{-1}s^{-2}$					&	$m^{-4}$					\\
			能量流密度$S$					&	$kg^{1}s^{-3}$							&	$m^{-4}$					\\
			能动量张量分量$T_{\mu\nu}$		&	$m^{-2}$								&	$m^{-2}$					\\
			Einstein张量分量$G_{\mu\nu}$	&	$m^{-2}$								&	$m^{-2}$					\\
			\midrule
			电磁学量						&				国际单位制					&	自然单位制	\\
			\midrule
			电流$I$							&	$A^{1}$									&	 $m^{-1}$					\\
			电场强度$E$						&	$A^{-1}kg^{1}m^{1}s^{-3}$				&	 $m^{-2}$					\\
			%电位移矢量$D$					&	$m^{1}$									&	 $m^{1}$					\\
			%磁场强度$H$					&	$m^{1}$									&	 $m^{1}$					\\
			磁感应强度$B$					&	$A^{-1}kg^{1}s^{-2}$					&	 $m^{-2}$					\\
			规范场分量$A_\mu$					&	$A^{-1}kg^{1}m^{1}s^{-2}$				&	 $m^{-1}$					\\
			场强张量分量$F_{\mu\nu}$		&	$A^{-1}kg^{1}s^{-2}$					&	 $m^{-2}$					\\
			\midrule
			热学量							&				国际单位制					&	自然单位制	\\
			\midrule
			温度$\varTheta$					&	$K^{1}$									&	 $m^{-1}$					\\
			$\cdots$						&	$\cdots$								&	 $\cdots$					\\




					
			\bottomrule
		\end{tabular*}
		\caption{常见物理量的单位换算关系}\label{tab:Dimensional}
		\end{table}