\section{Lagrange力学}\label{sec:Lagrange}
	
	\subsection{最小作用量原理}
		我们总是看到如“水自发的往低处流”,“光线的传播路径总是用时最少的路径”,或者“各种原子的核外电子排布总使得原子总能量最低”等观测现象.这些现象与数学中的“极值”概念相似,于是我们给出这样一个第一性原理,称为\textbf{最小作用量原理}.
		\begin{definition}
			客体的作用量取极值时,总能导出客体所满足的运动方程.
		\end{definition}
		这个定义云里雾里,什么是作用量?为什么作用量要取极值?我们给出这样的解释.
		
		作用量是这样一个无量纲的泛函
		\begin{equation}\label{eq:action1}
			S=\int ds,
		\end{equation}
		其中$ds$是某个微分式\footnote{$ds$本质上是一个微分形式场,即流形外形式丛的截面.我们这里只讨论$ds$是1微分形式场的情况,即普通的微分式.}.我们可以根据不同的情况分别考虑(\ref{eq:action1})的具体形式.令
		\begin{equation}\label{eq:action2}
			S(x^\mu(t);\dot{x}^\mu(t))=\int ds=\frac{1}{\alpha}\int L(x^\mu(t);\dot{x}^\mu(t))dt,
		\end{equation}
		其中$\alpha$是一个比例系数,$L(x^\mu(t);\dot{x}^\mu(t))$是一个函数,$t$是一个参数,这就是一般情形下的作用量.我们可以为式中的比例系数、函数以及参数赋予物理意义,使其拥有表征物理量的量纲,这样我们就可以使作用量描述物理系统了.例如,设比例系数$\alpha$具有长度的量纲,将函数$L$解释为折射率$L=n(x)$,并将参数$t$诠释为光走的几何路程$t=l$,此时(\ref{eq:action2})就变成了
		$$S=\int ds=\frac{1}{\alpha}\int n(x)dl,$$
		这样构造的作用量描述的就是光程.等等.

		于是我们就解释了什么是作用量.而对于“为什么作用量要取极值”这个问题,很遗憾,我无法回答.最小作用量原理是一个第一性原理,它无法被证伪.但我们知道,物理学是一门以实验为基础的学科.我们用最小作用量原理可以导出各种系统的运动方程,而这些运动方程预言的物理现象经受住了大量实验的考验.凭借“证有不证无”的观念,如果要证明最小作用量原理是错的,就要求我们拿出一个不满足最小作用量原理预言的实验证据,我们便可以拿这一实验证据去证明最小作用量原理的错误,但在此之前,我们将全盘接受这一原理.

		
		